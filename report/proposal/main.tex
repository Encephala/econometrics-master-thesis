% Template from https://www.overleaf.com/latex/templates/msc-thesis-template-erasmus-school-of-economics/fxsvshrthnqc
\documentclass[a4paper,11pt]{report}

\author{Jonathan Rietveld}
\title{Proposal Master Thesis}
% \date{An optional custom date, the default is today}
\newcommand{\studentnumber}{666788}
\newcommand{\program}{Business Analytics and Quantitative Marketing}
\newcommand{\supervisor}{Paul Bouman}
% \newcommand{\secondassesor}{Name of your second assessor}

\usepackage[british]{babel} % Use British English
\usepackage[onehalfspacing]{setspace} % Increase line spacing
\usepackage[margin=2.5cm]{geometry} % Modify margins
\usepackage{graphicx,booktabs} % Packages for images, tables, and APA citations

\usepackage{amsmath,amsfonts}
\usepackage[hidelinks]{hyperref}
\usepackage{cleveref,apacite}

\begin{document}

\begin{titlepage}
\makeatletter
\begin{center}
	\textsc{Erasmus University Rotterdam}
	\par \textsc{Erasmus School of Economics}
	\par Master Thesis \program

	\vfill \hrule height .08em \bigskip
	\par\huge\@title\bigskip
	\par\Large\@author\,(\studentnumber)\bigskip
	\hrule height .08em\normalsize

	\vfill
	\includegraphics[width=\textwidth,height=0.15\textheight,keepaspectratio]{../eur}
	\vfill

	\begin{tabular}{ll}
		\toprule
		Supervisor: & \supervisor\\
		% Second assessor: & \secondassesor\\
		Date final version: & \@date\\
		\bottomrule
	\end{tabular}

	\vfill
	% The content of this thesis is the sole responsibility of the author and does not reflect the view of the supervisor, second assessor, Erasmus School of Economics or Erasmus University.
\end{center}
\makeatother
\end{titlepage}

The impact of Physical Activity (PA) on mental wellbeing has been studied extensively in recent literature,
both in empirical studies \cite{noetel2024effect, mahindru2023role}, and through mechanistic arguments \cite{smith2021role}.
However, while cross-sectional studies consistently find a strong association between the two, \citeA{chekroud2018association}
note that the causal effect of PA on mental health as studied in randomised control trials has been inconsistent.
As such, only tentative conclusions have been drawn, claiming PA "is probably [beneficial] for psychiatric diseases"
\cite{peluso2005physical}, "hold(s) promise in the treatment [...] of mental health conditions" \cite{smith2021role}, et cetera.

I posit that longitudinal observational studies to date have not been powered to draw conclusions about the causal effect
of PA on mental health in part because they have not explicitly modelled the reverse effect, namely that individuals with poor
mental health are less likely to engage in PA.
\citeA{azevedo2012bidirectional} and \citeA{jerstad2010prospective} find empirical evidence for this reverse relationship,
though other work is inconclusive \cite{birkeland2009longitudinal, ku2012physical}. Nevertheless, combined with the
mechanistic argument that exercise increases self-efficacy \cite{smith2021role}, the assumption that engagement in PA is
not influenced by mental health seems tenuous at best.
A violation of this assumption leads to endogeneity when regressing mental health on PA and therethrough to
inconsistent estimation of the causal effect, a fact that is often neglected in longitudinal research \cite{leszczensky2022deal}.

% for instance chekroud(?)
% (also mention their numbers? contrast to numbers in RCT?)

% remedy the discrepancy between (cross-sectional) observational studies and RCT

% replicate results of chekroud
% then, use SEM to model simultaneity in LISS panel
% to demonstrate failure (better word maybe xdd) of existing longitudinal studies
% emphasis on within-person variation, to inform treatment decisions
% and contribute to literature in finding directionality of effect as well as temporal nature
% but panel so not very good for finding causal effect

\textit{Definition of the problem/question. What constitutes the problem? Which aspects are important?}

\textit{Relevance: Why and for whom is the research interesting and relevant? Is it of scientific relevance, and/or is it of interest for practical applications?}

\textit{Literature: What kind of results have been obtained in previous research on this topic? How does the research relate to the existing literature?}

\textit{Motivation: Why is the research necessary? Why is the existing knowledge on this topic insufficient? How will the research address these issues?}

\textit{Methods: Which (econometric) methods and techniques will be applied in your research? Why are the methods appropriate here?}

\textit{Data: What kind of data are needed and available for the research?}


\section*{Time frame}
[TODO]

\bibliographystyle{apacite}
\bibliography{references}

% \appendix

\end{document}
