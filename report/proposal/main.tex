% Template from https://www.overleaf.com/latex/templates/msc-thesis-template-erasmus-school-of-economics/fxsvshrthnqc
\documentclass[a4paper,11pt]{report}

\author{Jonathan Rietveld}
\title{Proposal Master Thesis}
% \date{An optional custom date, the default is today}
\newcommand{\studentnumber}{666788}
\newcommand{\program}{Business Analytics and Quantitative Marketing}
\newcommand{\supervisor}{Paul Bouman}
% \newcommand{\secondassesor}{Name of your second assessor}

\usepackage[british]{babel} % Use British English
\usepackage[onehalfspacing]{setspace} % Increase line spacing
\usepackage[margin=2.5cm]{geometry} % Modify margins
\usepackage{graphicx,booktabs} % Packages for images, tables, and APA citations

\usepackage{amsmath,amsfonts}
\usepackage[hidelinks]{hyperref}
\usepackage{cleveref,apacite}

\begin{document}

\begin{titlepage}
\makeatletter
\begin{center}
	\textsc{Erasmus University Rotterdam}
	\par \textsc{Erasmus School of Economics}
	\par Master Thesis \program

	\vfill \hrule height .08em \bigskip
	\par\huge\@title\bigskip
	\par\Large\@author\,(\studentnumber)\bigskip
	\hrule height .08em\normalsize

	\vfill
	\includegraphics[width=\textwidth,height=0.15\textheight,keepaspectratio]{../eur}
	\vfill

	\begin{tabular}{ll}
		\toprule
		Supervisor: & \supervisor\\
		% Second assessor: & \secondassesor\\
		Date final version: & \@date\\
		\bottomrule
	\end{tabular}

	\vfill
	% The content of this thesis is the sole responsibility of the author and does not reflect the view of the supervisor, second assessor, Erasmus School of Economics or Erasmus University.
\end{center}
\makeatother
\end{titlepage}

The impact of Physical Activity (PA) on mental wellbeing has been the topic of much recent literature,
both in empirical studies \cite{noetel2024effect, mahindru2023role}, and through mechanistic research \cite{smith2021role}.
However, while cross-sectional studies consistently find a strong association between the two, \citeA{chekroud2018association}
note that the causal effect of PA on mental health as studied in randomised control trials has been inconsistent.
\citeA{chalder2012facilitated} for instance find a highly insignificant change in the Beck's depression inventory score
of $-0.54$ ($p = 0.68$), while \citeA{philippot2022impact} find a signifiacnt decrease in the Hospital Anxiety and
Depression Scale of $3.8$ ($p = 0.016$).
Due to this inconsistency, review articles only draw tentative conclusions, f.i. that PA "is probably [beneficial]
for psychiatric diseases" \cite{peluso2005physical}, "hold(s) promise in the treatment [...] of mental health conditions"
\cite{smith2021role}, et cetera.

I posit that longitudinal observational studies to date have not been powered to draw conclusions about the causal effect
of PA on mental health, in part because they have not explicitly modelled the reverse effect, namely that individuals with poor
mental health are less likely to engage in PA.
\citeA{azevedo2012bidirectional} and \citeA{jerstad2010prospective} find empirical evidence for this reverse relationship,
though other work is inconclusive \cite{birkeland2009longitudinal, ku2012physical}. Nevertheless, combined with the
mechanistic argument that exercise increases self-efficacy \cite{smith2021role}, the assumption that engagement in PA is
not influenced by mental health seems tenuous at best.
A violation of this assumption leads to endogeneity when regressing mental health on PA and therethrough to
inconsistent estimation of the causal effect, a fact that \citeA{leszczensky2022deal} note is often neglected in
longitudinal research.

The aim of this work is then to remedy the discrepancy between on the one hand the consistent association found in
cross-sectional studies, and the inconsistent findings of trials on the other hand.
This is done by applying a statistical model that accomodates for the potentially mutual influence (simultaneity)
between PA and mental health. Inspired by the excellent statistical properties of Structural Equation Modelling (SEM)
in this context, namely its unbiasedness and significantly greater efficiency than Arellano-Bond (GMM) estimation
\cite{leszczensky2022deal}, the present study follows the procedures outlined by \citeA{allison2017maximum}
for modelling reciprocal causation.
SEM allows for modelling arbitrary (linear) interactions between variables, including dynamic panel regressions,
which makes it possible to flexibly compare model specifications through e.g. likelihood ratio tests.
SEM thus lends itself well to a statistical study of the relationships between variables.
Data from the LISS panel is studied, which is a representative (invite-only) panel of $7500$ Dutch individuals aged
sixteen and above \cite{scherpenzeel2010liss}. The panel comprises a broad range of questions on the topics of PA and
mental wellbeing, but also manifold other variables that can therefore be controlled for in the analysis.
Because the panel follows individuals for multiple years, it provides insight into the temporal nature of interactions
between PA and mental health.
In order to demonstrate the importance of properly modelling reverse causality, I first aim to closely follow
the analysis done by \citeA{chekroud2018association} to corroborate their main result that exercising decreases incidence
of poor mental health by $43\%$.
Then, the same question will be analysed in the SEM-framework to determine how much of this association can be attributed
to a causal effect. However, due to the measurement error in questionnaire responses, this study will not be powered
to accurately estimate the exact effect size \cite{pereira2014depressive}. As such, it is early to draw practical conclusions, but the data should nevertheless
be sufficient to find significant effects and to study if the potential reverse causality appreciably influences
parameter estimates.

A major challenge in doing so will be to find the appropriate lag structure, as the validity of the results heavily
relies on the correct specification of the lag structure. However, not only is this structure not known a priori,
it is also likely in general that the temporal effect of the predictor on the outcome does not align with the sampling
frequency (i.e. the time interval between panel waves) and the present literature has not established how SEM modelling
may then be effectively applied \cite{leszczensky2022deal}.
% MAJOR TODO: Discuss complexities of missing data
It is additionally important to appropriately control for individual-specific effects, as for instance physical
health makes a significant confounder, as it may preclude physical activity while also having a well-established impact
on mental health \cite{ohrnberger2017relationship}, thus leading to endogeneity. However, as panel regressions naturally
allow for controlling for individual-specific effects, that problem is readily solved.

The present study aims to guide further research, both through contributing to a better understanding of the potentially
complicated interdependence between mental health and PA, as well as by establishing the importance of considering this
interdependence when studying the topic.

\pagebreak
\section*{Time frame}
I aim to finish the initial analysis halfway through March, working out practical difficulties in working with the LISS
panel data in the process.
Subsequently developing the SEM model will take much longer, as that is a more complicated model that relies on more
assumptions and will be more heavily influenced by missing data.
I hope to be done with it during the course of May, leaving the rest of May and June, perhaps running into early July,
for writing and integrating feedback. I thus aim to submit the thesis around the end of June or start of July.
If the process of developping the SEM model goes smoother than expected, I might use the freed up time to expand the study,
for instance by considering the moderating role of the type of exercise, or studying the mechanisms for the effect of PA
on mental health in a mediation analysis.

\bibliographystyle{apacite}
\bibliography{references}

% \appendix

\end{document}
