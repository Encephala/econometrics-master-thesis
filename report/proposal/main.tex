% Template from https://www.overleaf.com/latex/templates/msc-thesis-template-erasmus-school-of-economics/fxsvshrthnqc
\documentclass[a4paper,11pt]{report}

\author{Jonathan Rietveld}
\title{Proposal Master Thesis}
% \date{An optional custom date, the default is today}
\newcommand{\studentnumber}{666788}
\newcommand{\program}{Business Analytics and Quantitative Marketing}
\newcommand{\supervisor}{Paul Bouman}
% \newcommand{\secondassesor}{Name of your second assessor}

\usepackage[british]{babel} % Use British English
\usepackage[onehalfspacing]{setspace} % Increase line spacing
\usepackage[margin=2.5cm]{geometry} % Modify margins
\usepackage{graphicx,booktabs} % Packages for images, tables, and APA citations

\usepackage{amsmath,amsfonts}
\usepackage[hidelinks]{hyperref}
\usepackage{cleveref,apacite}

\begin{document}

\begin{titlepage}
\makeatletter
\begin{center}
	\textsc{Erasmus University Rotterdam}
	\par \textsc{Erasmus School of Economics}
	\par Master Thesis \program

	\vfill \hrule height .08em \bigskip
	\par\huge\@title\bigskip
	\par\Large\@author\,(\studentnumber)\bigskip
	\hrule height .08em\normalsize

	\vfill
	\includegraphics[width=\textwidth,height=0.15\textheight,keepaspectratio]{../eur}
	\vfill

	\begin{tabular}{ll}
		\toprule
		Supervisor: & \supervisor\\
		% Second assessor: & \secondassesor\\
		Date final version: & \@date\\
		\bottomrule
	\end{tabular}

	\vfill
	% The content of this thesis is the sole responsibility of the author and does not reflect the view of the supervisor, second assessor, Erasmus School of Economics or Erasmus University.
\end{center}
\makeatother
\end{titlepage}

The impact of Physical Activity (PA) on mental wellbeing has been studied extensively in recent literature,
both in empirical studies \cite{noetel2024effect, mahindru2023role}, and through mechanistic arguments \cite{smith2021role}.
However, while cross-sectional studies consistently find a strong association between the two, \citeA{chekroud2018association}
note that the causal effect of PA on mental health as studied in randomised control trials has been inconsistent.
As such, only tentative conclusions have been drawn, claiming PA "is probably [beneficial] for psychiatric diseases"
\cite{peluso2005physical}, "hold(s) promise in the treatment [...] of mental health conditions" \cite{smith2021role}, et cetera.
% TODO: example numbers for the strength of the association to compare correlation vs (potential) causation
% Also noting effect sizes have been found to be comparable with drugs, if possible

I posit that longitudinal observational studies to date have not been powered to draw conclusions about the causal effect
of PA on mental health in part because they have not explicitly modelled the reverse effect, namely that individuals with poor
mental health are less likely to engage in PA.
\citeA{azevedo2012bidirectional} and \citeA{jerstad2010prospective} find empirical evidence for this reverse relationship,
though other work is inconclusive \cite{birkeland2009longitudinal, ku2012physical}. Nevertheless, combined with the
mechanistic argument that exercise increases self-efficacy \cite{smith2021role}, the assumption that engagement in PA is
not influenced by mental health seems tenuous at best.
A violation of this assumption leads to endogeneity when regressing mental health on PA and therethrough to
inconsistent estimation of the causal effect, a fact that is often neglected in longitudinal research \cite{leszczensky2022deal}.

The aim of this work is then to remedy the discrepancy between the consistent association found in cross-sectional studies
on the one hand, and the inconsistent findings of trials on the other hand, by applying a statistical model that accomodates
for the potentially mutual influence (simultaneity) between PA and mental health. Inspired by the excellent statistical
properties found by \citeA{leszczensky2022deal} of Structural Equation Modelling (SEM) in this context, namely
unbiasedness with significantly greater efficiency than Arellano-Bond (GMM) estimation, the present study
follows the procedures outlined by \citeA{allison2017maximum} for modelling reciprocal causation.
Data from the LISS panel is studied, which is a representative (invite-only) panel of $7500$ dutch individuals
aged sixteen and above \cite{scherpenzeel2010liss}.
In order to demonstrate the importance of incorporating reverse causality in modelling, I first aim to roughly reproduce
the main result of \citeA{chekroud2018association}, namely that exercising decreases incidence of poor mental health by
$43\%$.
Next, the same question will be analysed in the SEM-framework to distill which part of this association that is a causal effect.
However, due to the measurement error in questionnaire responses, this study will not be powered to accurately estimate
the exact effect size. Nevertheless, the data should be sufficient to find directionality of effects and ballpark
effect sizes \cite{pereira2014depressive}.
A major challenge in doing so will be to find the appropriate lag structure. This because the validity of the results heavily
relies on the correct model specification in that sense, while it is likely in general that the temporal effect
of the predictor on the outcome does not align with the sampling frequency (i.e. the time interval between panel waves)
and only limited research is available on this scenario \cite{leszczensky2022deal}.

The aim of this study is to best inform the question of whether PA can be employed as a treatment for mental health conditions.
This guides the emphasis of the analysis in two ways;
firstly, the "forward" effect of PA on mental health is emphasised, while the reverse effect is considered a nuisance parameter,
following \citeA{allison2017maximum};
secondly, through employing panel data, the within-person change in mental health attributable to PA is emphasised,
as opposed to the between-person association that cross-sectional data studies.


\textit{Definition of the problem/question. What constitutes the problem? Which aspects are important?}

\textit{Relevance: Why and for whom is the research interesting and relevant? Is it of scientific relevance, and/or is it of interest for practical applications?}

\textit{Literature: What kind of results have been obtained in previous research on this topic? How does the research relate to the existing literature?}

\textit{Motivation: Why is the research necessary? Why is the existing knowledge on this topic insufficient? How will the research address these issues?}

\textit{Methods: Which (econometric) methods and techniques will be applied in your research? Why are the methods appropriate here?}

\textit{Data: What kind of data are needed and available for the research?}


\section*{Time frame}
[TODO]

\bibliographystyle{apacite}
\bibliography{references}

% \appendix

\end{document}
