Much recent literature aimed to find interventions to improve mental wellbeing, including physical exercise.
However, such a benefit for exercise has not been well established, with especially trials having inconsistent findings
in contrast with the positive correlation between exercise and mental wellbeing that is consistently found in observational
studies.
The present study aims to find a causal benefit in observational data through careful analysis of data from the
LISS panel, a yearly panel in the Netherlands with 7500 participants.
It is hypothesised that the discrepancy between trials and observational analysis can be attributed
to the reverse action of poor mental wellbeing making one less likely to engage in exercise.
To this end, only the effect of previous exercise on present wellbeing is considered, which could not be compromised
by such reverse causality.
The main finding is that there is no significant effect of previous exercise on mental wellbeing ($p = 0.095$).
This does not preclude an appreciable short-term benefit of exercising, warranting further research on the topic.
Such research should emphasise controlling for reverse causality in the experimental design, as that appears
the only feasible way to differentiate the short-term effect from the reverse action.
