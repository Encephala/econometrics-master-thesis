For the definitive models in \textit{lavaan} syntax, refer to the programming code (\cref{chap:app:code}).

\section{Excluding Mediation}
\label{sec:results:no_mediation}

To reiterate, in this approach mediation regressions are not included and neither are the mediators included as regressors,
gaining model parsimony at risk of violating the ignorability assumption.
The regression parameters are listed in \cref{tab:results:basic_regression}. Since besides the autoregressive
terms each variable included is a binary variable, the parameters can be interpreted as the percentage point change
in mental wellbeing (MHI5) as a result of that variable.
In general, very few estimated parameters are significant, and even those that are represent only a marginal impact,
no greater than $1.4$ points for an income above € 50\,000 as compared to no income. Contrast this with the interquartile
range in MHI5 of about $20$ (\cref{fig:data:sample_moments_y_x}).

% TODO: Discuss fit indices

The results may especially be put into question for the very small instantaneous association between sports and MHI5.
While statistically significant, an average improvement of $0.48$ on a scale of 0 to 100 for individuals who exercise
is in stark contrast with cross-sectional analyses in the literature. Recall for instance the 43\% decrease in days of poor mental
health found in \citeA{chekroud2018association}, or the review by \citeA{noetel2024effect} who find effect sizes of between
$0.2$ and $0.8$ times the sample standard deviation on various measures.
The estimated long-term effect of exercise is highly insignificant ($p = 0.888$), and hence the null hypothesis that
no such long-term benefit exists is maintained.

In time series analysis, it is customary to examine the stability of the data, as stability is typically a necessary
assumption for estimation. While SEM does not rely explicitly on this assumption, absence of stability would at least make the
assumption that parameters are constant in time contentious, and the derivation of the long-term effect as a change
in equilibrium value (\cref{eq:methods:long_run_effect}) does rely on stability.
Based on the reported autoregressive coefficients, the roots of the characteristic polynomial are $z_1 = 1.10$,
$z_2 = -1.16 + 1.78i$ and $z_3 = -1.16 - 1.78i$, which all fall decidedly outside of the unit circle,
indeed indicating a stable time series.

\newcolumntype{L}{>{\hspace*{3mm}}l}
\begin{table}[htbp]
    \centering
    \caption{Parameter estimates and fit indices for the base regression.
    Estimates are changes in mean MHI5-scores with respect to the dummy level in parentheses.
    Fit indices are robust variants where applicable}
    \label{tab:results:basic_regression}
    \begin{tabular}{
        L
        S[table-format=2.3] % three decimals, two digits before
        S[table-format=2.3]
        S[table-format=2.3]
        S[table-format=1.3]
    }
    \toprule

    \textbf{Regressor} & \textbf{Estimate} & \textbf{Std. Error} & \textbf{z-value} & \textbf{p-value} \\

    \midrule

    MHI5$_{t-1}$                    & 0.397     & 0.007 & 54.254    & 0.000 *** \\
    MHI5$_{t-2}$                    & 0.248     & 0.008 & 30.737    & 0.000 *** \\
    MHI5$_{t-3}$                    & 0.202     & 0.007 & 26.939    & 0.000 *** \\

    sports$_t$                      & 0.432     & 0.177 & 2.746     & 0.013 * \\
    sports$_{t-1}$                  & -0.025    & 0.177 & -0.140    & 0.888 \\

    \multicolumn{5}{l}{\textit{age} (below 18 years)} \\
    18-24 years                     & -0.343    & 0.372 & -0.923    & 0.356 \\
    25-39 years                     & -0.107    & 0.438 & -0.244    & 0.807 \\
    40-66 years                     & 0.700     & 0.443 & 1.579     & 0.114 \\
    over 67 years                   & 0.331     & 0.490 & 0.676     & 0.499 \\

    \multicolumn{5}{l}{\textit{income} (none)} \\
    below € 15\,000                 & 1.042     & 0.536 & 1.944     & 0.052 $^+$ \\
    over € 50\,000                  & 1.331     & 1.178 & 1.129     & 0.259 \\

    \multicolumn{5}{l}{\textit{immigration status} (Dutch)} \\
    first generation western        & -0.688    & 0.257 & -2.683    & 0.007 ** \\
    first generation non-western    & -0.959    & 0.268 & -3.576    & 0.000 *** \\
    second generation western       & -0.403    & 0.223 & -1.809    & 0.070 $^+$ \\
    second generation non-western   & -0.245    & 0.277 & -0.885    & 0.376 \\

    \multicolumn{5}{l}{\textit{gender} (female)} \\
    male                            & 0.562     & 0.103 & 5.474     & 0.000 *** \\

    \multicolumn{5}{l}{\textit{marital status} (divorced)} \\
    married                         & 0.155     & 0.178 & 0.873     & 0.383 \\
    never been married              & -0.651    & 0.207 & -3.139    & 0.002 ** \\
    separated                       & -0.813    & 0.859 & -0.947    & 0.344 \\
    widow or widower                & 0.337     & 0.289 & 1.167     & 0.243 \\

    \multicolumn{5}{l}{\textit{education level} (havo-vwo)} \\
    hbo                             & -0.176    & 0.193 & -0.914    & 0.361 \\
    mbo                             & 0.021     & 0.196 & 0.110     & 0.913 \\
    primary school                  & -0.382    & 0.273 & -1.401    & 0.161 \\
    vmbo                            & -0.056    & 0.205 & -0.273    & 0.785 \\
    university (wo)                 & -0.436    & 0.218 & -0.273    & 0.046 * \\

    \multicolumn{5}{l}{\textit{employment status} (employed)} \\
    homemaker                       & -0.510    & 0.195 & -2.618    & 0.009 ** \\
    retired                         & 0.063     & 0.191 & 0.327     & 0.744 \\
    student                         & -0.594    & 0.337 & -1.764    & 0.078 $^+$ \\
    unable to work                  & -1.409    & 0.313 & -4.505    & 0.000 *** \\
    unemployed                      & -0.415    & 0.365 & -1.137    & 0.256 \\

    \midrule

    Observations    & \multicolumn{4}{l}{12920} \\
    Parameters      & \multicolumn{4}{l}{TODO (TODO equality constraints)} \\
    $\chi^2$        & \multicolumn{4}{l}{1530.6 ($df = 270$, $p = 0.000$)} \\
    CFI             & \multicolumn{4}{l}{0.956 (cutoff = 0.95)} \\
    TLI             & \multicolumn{4}{l}{0.951 (cutoff = 0.95)} \\
    RMSEA           & \multicolumn{4}{l}{0.019 (cutoff = 0.06)} \\
    SRMR            & \multicolumn{4}{l}{0.026 (cutoff = 0.08)} \\

    \bottomrule

    \multicolumn{5}{l}{\textit{Significance levels:} $^+$ 0.10, * 0.05, ** 0.01, *** 0.001} \\
\end{tabular}
\end{table}

\section{Including Mediation}
\label{sec:results:mediation}

The main findings of interest in the mediation analysis are to what extent the effect is explained by the mediators,
and the degree to which the mediators confounded the result as indicated by the change in the estimated total effect.
The model contains an overwhelming number of parameters, which would be difficult to interpret.
Hence, \cref{tab:results:mediation_regression} lists the main parameters of interest, namely those of sports
in the mediation regressions and those of sports and the mediators in the main regression.
For the complete parameter estimates, refer to \cref{chap:app:mediation_regression}.

% TODO update with numbers with lagged mediators-numbers!
The instantaneous effect of sports as well as the (direct) lagged effect are now notably more negative than when no mediators
are included, changing by $-0.390 \pm 0.24$ and $-0.142 \pm 0.25$ respectively. This points towards there indeed being a
positive effect through the mediators, with the large estimated effects of physical health showing it especially is an important
factor. The change in the instantaneous association also provides some evidence for mediation of the reverse
action, i.e. that being sick or being in poor physical health decreases chance to exercise, although the difference
is not significant ($p = 0.110$ based on the above numbers).

Being ill decreases mental wellbeing ($p < 0.001$), but only moderately with a $-1.384$ percentage point change
in MHI5 on average.
Interesting is that being being ill last year has a similarly large but positive effect of $1.474$ percentage points,
indicating that having recovered from a disease improves wellbeing and/or that having just fallen ill is associated
with a larger mental burden than having been ill for a longer time.
We do not find a significant long-term effect of exercise on disease status ($p = 0.155$).
This indicates that exercise does not seem to strengthen the immune system in the long run,
and because the LISS panel also queries bone fractures, exercise may also not have a beneficial effect towards
strengthening the skeleton, regardless of for instance \citeA{hong2018effects}'s findings that exercise
increases bone mass.
However, because this includes various forms of disease simultaneously, the finding does not preclude a positive effect of exercise on
some forms of disease as in \citeA{westcott2012resistance}, if those effects are offset with a negative effect on other forms
of disease.
In total, the causal long-term effect of physical exercise on mental wellbeing as moderated by disease is $-0.009$
percentage points, which is statistically insignificant at $p = 0.168$.

Physical health is found to have by far the greatest impact on MHI5 of any regressor,
with effect sizes as high as $-19.840$ percentage points.
There is a clear trend that worse physical health is associated with worse mental health.
Strikingly however, the lagged effects all have an opposite sign, that is, a positive association with mental wellbeing.
This is evidence of a similar rebound mechanism as with disease, where having recently had physical health improve is
cause for happiness, and vice versa.
Exercise generally relates to better physical health, but effect sizes are modest, with the greatest change found
being that exercise is associated with a 4.0\% increased probability of being in good physical health.
Lagged exercise significantly decreases probability of moderate physical health ($-2.4\%$, $p < 0.001$) while increasing
the probability of very good mental health ($2.3\%$, $p < 0.001$).
Interesting is the finding that the probabilities of reporting poor or excellent health
are only weakly influenced by instantaneous and lagged sports, indicating that at the extremes of physical health,
the effects of sports are overwhelmed by the effects of other (unmeasured) factors.
Combined with the large detrimental effect of last year's physical health on MHI5, we find a causal effect
of $-0.117 \pm 0.022$ percentage points MHI5 ($p < 0.001$). The negative sign of this effect is contrary to expectations based on the
literature, which find a positive effect if any. This oddity is due to the aforementioned rebound effect.

% TODO BMI

Combining the direct effect and the effect through the mediators, we conclude that engaging in exercise \textit{increases/decreases}
MHI5 on average by \textit{TODO}, equivalent to a long-run effect of \textit{TODO * 4.55}.
% TODO compare to nonmediation

% TODO largest effect size in all the mediation regressions, not just in the variables of interest?
% or justify that the approximation is most relevant among the variables of interest
The largest effect size found in mediation regressions is $0.044$, which is quite small, justifying the approximation
of using linear models over a logit analysis.
% TODO discuss parallel mediation

\begin{table}[htbp]
    \scriptsize
    \centering
    \caption{Parameters of interest in the mediation analysis.
    Regressands are in bold.
    Standard errors in total effect are determined by the delta method}
    \label{tab:results:mediation_regression}
    \begin{tabular}{
        L
        S[table-format=2.3]
        S[table-format=2.3]
        S[table-format=2.3]
        S[table-format=1.3]
    }
    \toprule

    \textbf{Regressor} & \textbf{Estimate} & \textbf{Std. Error} & \textbf{z-value} & \textbf{p-value} \\

    \midrule

    \multicolumn{5}{l}{\textbf{MHI5$_t$}} \\
    MHI5$_{t-1}$                    & 0.391     & 0.007 & 54.301    & 0.000 *** \\
    MHI5$_{t-2}$                    & 0.227     & 0.008 & 29.093    & 0.000 *** \\
    MHI5$_{t-3}$                    & 0.182     & 0.007 & 25.401    & 0.000 *** \\

    sports$_t$                      & 0.042     & 0.168 & 0.249     & 0.803 \\
    sports$_{t-1}$                  & -0.167    & 0.172 & -0.969    & 0.332 \\

    disease status$_t$              & -1.384    & 0.216 & -6.398    & 0.000 *** \\
    disease status$_{t-1}$          & 1.474     & 0.221 & 6.677     & 0.000 *** \\

    \multicolumn{5}{l}{\textit{physical health}$_t$ (excellent)} \\
    good                            & -6.366    & 0.310 & -20.566   & 0.000 *** \\
    moderate                        & -12.818   & 0.402 & -31.907   & 0.000 *** \\
    poor                            & -19.840   & 0.861 & -23.050   & 0.000 *** \\
    very good                       & -2.785    & 0.283 & -9.859    & 0.000 *** \\
    \multicolumn{5}{l}{\textit{physical health}$_{t-1}$ (excellent)} \\
    good                            & 2.952     & 0.342 & 8.631     & 0.000 *** \\
    moderate                        & 6.176     & 0.422 & 14.642    & 0.000 *** \\
    poor                            & 10.746    & 0.820 & 13.102    & 0.000 *** \\
    very good                       & 1.694     & 0.320 & 5.288     & 0.000 *** \\

    \multicolumn{5}{l}{\textit{BMI}$_t$ (underweight)} \\
    normal weight                   & 0.306     & 0.518 & 0.591     & 0.554 \\
    overweight                      & 1.241     & 0.552 & 2.247     & 0.025 * \\
    obese                           & 1.232     & 0.603 & 2.042     & 0.041 * \\
    \multicolumn{5}{l}{\textit{BMI}$_{t-1}$ (underweight)} \\
    normal weight                   & 0.347     & 0.567 & 0.612     & 0.541 \\
    overweight                      & -0.020    & 0.602 & -0.033    & 0.973 \\
    obese                           & 0.207     & 0.656 & 0.315     & 0.752 \\

    \midrule

    \multicolumn{5}{l}{\textbf{Disease status$_t$}} \\
    sports$_t$                      & -0.007    & 0.004 & -1.608    & 0.108 \\
    sports$_{t-1}$                  & -0.006    & 0.004 & -1.423    & 0.155 \\

    \midrule

    \multicolumn{5}{l}{\textbf{Physical health$_t$} (excellent)} \\
    good                            & \multicolumn{4}{l}{} \\
    \hspace{3mm} sports$_t$         & -0.009    & 0.007 & -1.400    & 0.162 \\
    \hspace{3mm} sports$_{t-1}$     & -0.009    & 0.007 & -1.400    & 0.162 \\

    moderate                        & \multicolumn{4}{l}{} \\
    \hspace{3mm} sports$_t$         & -0.020    & 0.007 & -3.007    & 0.003 ** \\
    \hspace{3mm} sports$_{t-1}$     & -0.024    & 0.005 & -4.906    & 0.000 *** \\

    poor                            & \multicolumn{4}{l}{} \\
    \hspace{3mm} sports$_t$         & -0.007    & 0.002 & -4.610    & 0.000 *** \\
    \hspace{3mm} sports$_{t-1}$     & 0.001     & 0.002 & 0.957     & 0.339 \\

    very good                       & \multicolumn{4}{l}{} \\
    \hspace{3mm} sports$_t$         & 0.040     & 0.005 & 7.810     & 0.000 *** \\
    \hspace{3mm} sports$_{t-1}$     & 0.023     & 0.005 & 4.422     & 0.000 *** \\

    \midrule

    \multicolumn{5}{l}{\textbf{BMI$_t$} (underweight)} \\
    normal weight                   & \multicolumn{4}{l}{} \\
    \hspace{3mm} sports$_t$         & 0.017     & 0.004 & 4.011     & 0.009 ** \\
    \hspace{3mm} sports$_{t-1}$     & 0.009     & 0.005 & 2.075     & 0.038 * \\

    overweight                      & \multicolumn{4}{l}{} \\
    \hspace{3mm} sports$_t$         & 0.003     & 0.005 & 0.632     & 0.527 \\
    \hspace{3mm} sports$_{t-1}$     & -0.002    & 0.005 & -0.329    & 0.742 \\

    obese                           & \multicolumn{4}{l}{} \\
    \hspace{3mm} sports$_t$         & -0.004    & 0.003 & -1.372    & 0.170 *** \\
    \hspace{3mm} sports$_{t-1}$     & -0.004    & 0.003 & -1.372    & 0.170 \\

    \midrule

    Observations    & \multicolumn{4}{l}{12920} \\
    Parameters      & \multicolumn{4}{l}{3100 (1891 equality constraints)} \\
    $\chi^2$        & \multicolumn{4}{l}{52855.2 ($df = 5061$, $p = 0.000$)} \\
    CFI             & \multicolumn{4}{l}{0.864 (cutoff = 0.95)} \\
    TLI             & \multicolumn{4}{l}{0.835 (cutoff = 0.95)} \\
    RMSEA           & \multicolumn{4}{l}{0.027 (cutoff = 0.06)} \\
    SRMR            & \multicolumn{4}{l}{0.089 (cutoff = 0.08)} \\

    \bottomrule

    \multicolumn{5}{l}{\textit{Significance levels:} $^+$ 0.10, * 0.05, ** 0.01, *** 0.001} \\
    \end{tabular}
\end{table}

\begin{table}
    \centering
    \caption{Effect through each mediator and direct effect, as derived from \cref{tab:results:mediation_regression}}
    \label{tab:results:mediation_total_effect}
    \begin{tabular}{
        L
        S[table-format=2.3]
        S[table-format=2.3]
        S[table-format=2.3]
        S[table-format=1.3]
    }

    \toprule

    \textbf{Regressor} & \textbf{Estimate} & \textbf{Std. Error} & \textbf{z-value} & \textbf{p-value} \\

    \midrule

    Direct effect                   & -0.167    & 0.172 & -0.969    & 0.332 \\
    Effect disease status           & -0.009    & 0.007 & -1.380    & 0.168 \\
    Effect physical health          & -0.117    & 0.022 & -5.399    & 0.000 *** \\
    Effect BMI                      & 0.002     & 0.003 & 0.762     & 0.761 \\
    Total effect                    & -0.290    & 0.174 & -1.671    & 0.095 \\

    \bottomrule

    \multicolumn{5}{l}{\textit{Significance levels:} $^+$ 0.10, * 0.05, ** 0.01, *** 0.001} \\
    \end{tabular}
\end{table}

\section{Discussion and Future Research}
\label{sec:results:discussion}
% TODO: make good text go be good
There are two competing interpretations of the results. First, the null hypothesis that there is no long-term effect of
exercise on mental health cannot be rejected based on the data.
Alternatively, neither can the null hypothesis that observational data does not provide sufficient evidence to
answer the research question. Both interpretations will now be discussed.

% Long term effect
% Implies: mechanisms that don't have significant impact (relate to mediation estimates), mechanisms that do
% Consistent with some RCT, although RCT only measure short-term effect I suppose

In this sense, the present study is consistent with the outcomes
of randomised controlled trials in the literature, many of which also do not find evidence for a beneficial effect
(e.g. \citeA{chalder2012facilitated}).

% Analysis impossible
% Major impacts: Lags due to reverse causality (long term effect likely much shorter). Would be solved by RCT with DID design
% Lesser impacts: data quality (MNAR, self-reporting). They bias effect towards zero, but not big enough to eliminate
% Find citation for size of bias due to self-report, reiterate number of MNAR

% TODO future research angles based on the interpretation
% Both interpretations inform choice for RCT, emphasis on not self-reported data and avoiding MNAR
% long-run harder, perhaps DID
% but long-run perhaps just observational data, at least prospective questionnaire, be satisfied with Granger
% causality combined with mechanistic arguments
