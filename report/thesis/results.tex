For the definitive models in \textit{lavaan} syntax, refer to the programming code (\cref{chap:app:code}).

\section{Excluding Mediation}
\label{sec:results:no_mediation}

To reiterate, in this approach mediation regressions are not included and neither are the mediators included as regressors,
gaining model parsimony at risk of violating the ignorability assumption.
The regression parameters are listed in \cref{tab:results:basic_regression}. Since besides the autoregressive
terms each variable included is a binary variable, the parameters can be interpreted as the percentage point change
in mental wellbeing (MHI5) as a result of that variable.
In general, very few estimated parameters are significant, and even those that are represent only a marginal impact,
no greater than $1.4$ points for an income above € 50\,000 as compared to no income. Contrast this with the interquartile
range in MHI5 of about $20$ (\cref{fig:data:sample_moments_y_x}).

% TODO: Discuss fit indices

The results may especially be put into question for the very small instantaneous association between sports and MHI5.
While statistically significant, an average improvement of $0.48$ on a scale of 0 to 100 for individuals who exercise
is in stark contrast with cross-sectional analyses in the literature. Recall for instance the 43\% decrease in days of poor mental
health found in \citeA{chekroud2018association}, or the review by \citeA{noetel2024effect} who find effect sizes of between
$0.2$ and $0.8$ times the sample standard deviation on various measures.
The estimated long-term effect of exercise is highly insignificant ($p = 0.888$), and hence the null hypothesis that
no such long-term benefit exists is maintained.

In time series analysis, it is customary to examine the stability of the data, as stability is typically a necessary
assumption for estimation. While SEM does not rely explicitly on this assumption, absence of stability would at least make the
assumption that parameters are constant in time contentious, and the derivation of the long-term effect as a change
in equilibrium value (\cref{eq:methods:long_run_effect}) does rely on stability.
Based on the reported autoregressive coefficients, the roots of the characteristic polynomial are $z_1 = 1.10$,
$z_2 = -1.16 + 1.78i$ and $z_3 = -1.16 - 1.78i$, which all fall decidedly outside of the unit circle,
indeed indicating a stable time series.

\newcolumntype{L}{>{\hspace*{3mm}}l}
\begin{table}[htbp]
    \centering
    \caption{Parameter estimates and fit indices for the base regression.
    Estimates are changes in mean MHI5-scores with respect to the dummy level in parentheses.
    Fit indices are robust variants where applicable}
    \label{tab:results:basic_regression}
    \begin{tabular}{
        L
        S[table-format=2.3] % three decimals, two digits before
        S[table-format=2.3]
        S[table-format=2.3]
        S[table-format=1.3]
    }
    \toprule

    \textbf{Regressor} & \textbf{Estimate} & \textbf{Std. Error} & \textbf{z-value} & \textbf{p-value} \\

    \midrule

    MHI5$_{t-1}$                    & 0.397     & 0.007 & 54.254    & 0.000 *** \\
    MHI5$_{t-2}$                    & 0.248     & 0.008 & 30.737    & 0.000 *** \\
    MHI5$_{t-3}$                    & 0.202     & 0.007 & 26.939    & 0.000 *** \\

    sports$_t$                      & 0.432     & 0.177 & 2.746     & 0.013 * \\
    sports$_{t-1}$                  & -0.025    & 0.177 & -0.140    & 0.888 \\

    \multicolumn{5}{l}{\textit{age} (below 18 years)} \\
    18-24 years                     & -0.343    & 0.372 & -0.923    & 0.356 \\
    25-39 years                     & -0.107    & 0.438 & -0.244    & 0.807 \\
    40-66 years                     & 0.700     & 0.443 & 1.579     & 0.114 \\
    over 67 years                   & 0.331     & 0.490 & 0.676     & 0.499 \\

    \multicolumn{5}{l}{\textit{income} (none)} \\
    below € 15\,000                 & 1.042     & 0.536 & 1.944     & 0.052 $^+$ \\
    over € 50\,000                  & 1.331     & 1.178 & 1.129     & 0.259 \\

    \multicolumn{5}{l}{\textit{immigration status} (Dutch)} \\
    first generation western        & -0.688    & 0.257 & -2.683    & 0.007 ** \\
    first generation non-western    & -0.959    & 0.268 & -3.576    & 0.000 *** \\
    second generation western       & -0.403    & 0.223 & -1.809    & 0.070 $^+$ \\
    second generation non-western   & -0.245    & 0.277 & -0.885    & 0.376 \\

    \multicolumn{5}{l}{\textit{gender} (female)} \\
    male                            & 0.562     & 0.103 & 5.474     & 0.000 *** \\

    \multicolumn{5}{l}{\textit{marital status} (divorced)} \\
    married                         & 0.155     & 0.178 & 0.873     & 0.383 \\
    never been married              & -0.651    & 0.207 & -3.139    & 0.002 ** \\
    separated                       & -0.813    & 0.859 & -0.947    & 0.344 \\
    widow or widower                & 0.337     & 0.289 & 1.167     & 0.243 \\

    \multicolumn{5}{l}{\textit{education level} (havo-vwo)} \\
    hbo                             & -0.176    & 0.193 & -0.914    & 0.361 \\
    mbo                             & 0.021     & 0.196 & 0.110     & 0.913 \\
    primary school                  & -0.382    & 0.273 & -1.401    & 0.161 \\
    vmbo                            & -0.056    & 0.205 & -0.273    & 0.785 \\
    university (wo)                 & -0.436    & 0.218 & -0.273    & 0.046 * \\

    \multicolumn{5}{l}{\textit{employment status} (employed)} \\
    homemaker                       & -0.510    & 0.195 & -2.618    & 0.009 ** \\
    retired                         & 0.063     & 0.191 & 0.327     & 0.744 \\
    student                         & -0.594    & 0.337 & -1.764    & 0.078 $^+$ \\
    unable to work                  & -1.409    & 0.313 & -4.505    & 0.000 *** \\
    unemployed                      & -0.415    & 0.365 & -1.137    & 0.256 \\

    \midrule

    Observations    & \multicolumn{4}{l}{12920} \\
    $\chi^2$        & \multicolumn{4}{l}{1530.6 ($df = 270$, $p = 0.000$)} \\
    CFI             & \multicolumn{4}{l}{0.956 (cutoff = 0.95)} \\
    TLI             & \multicolumn{4}{l}{0.951 (cutoff = 0.95)} \\
    RMSEA           & \multicolumn{4}{l}{0.019 (cutoff = 0.06)} \\
    SRMR            & \multicolumn{4}{l}{0.026 (cutoff = 0.08)} \\

    \bottomrule

    \multicolumn{5}{l}{\textit{Significance levels:} $^+$ 0.10, * 0.05, ** 0.01, *** 0.001} \\
\end{tabular}
\end{table}

\section{Including Mediation}
\label{sec:results:mediation}

The main findings of interest in the mediation analysis are to what extent the effect is explained by the mediators,
and the degree to which the mediators confounded the result as indicated by the change in the estimated total effect.
The model contains an overwhelming number of parameters, which would be difficult to interpret.
Hence, \cref{tab:results:mediation_regression} lists the main parameters of interest, namely those of sports
in the mediation regressions and those of sports and the mediators in the main regression.
For the complete parameter estimates, refer to \cref{chap:app:mediation_regression}.

% TODO update with numbers with lagged mediators-numbers!
The instantaneous effect of sports as well as the (direct) lagged effect are now notably more negative than when no mediators
are included, changing by $-0.455 \pm 0.24$ and $-0.278 \pm 0.25$ respectively. This indicates there is indeed a
positive effect through the mediators, with the large estimated effects of physical health showing it especially is an important
factor. The change in the instantaneous association also provides some evidence for mediation of the reverse
action, i.e. that being sick or being in poor physical health decreases chance to exercise, although the difference
is only weakly significant ($p = 0.057$).
Curious is the negative effect of sports last year of $-0.303$, although it is only weakly significant at $p = 0.08$,
so it should be interpreted with caution. Note that this estimate is controlled for disease status and physical health
that year, and for sports, disease status and physical health in the present year. A potential explanation may be that
failure to adhere to an exercise regime is in itself a cause for distress, although e.g. \citeA{sullins2019exercise}
do not find significant evidence for this association.

Disease status appears only of minor impact to mental wellbeing ($p = 0.061$),
and neither does it appear influenced significantly by exercise ($p = 0.188, 0.139$).
That being said, illness is of course strongly associated with physical health,
which is by far the most impactful variable in the present study, with the average difference in
MHI5 score of $14.008\,(p < 0.001)$ between individuals with poor and excellent physical health.
We also find some significant associations between sports and physical health, with the expected outcome that
exercise generally increases the probability to report higher levels of physical health.
Focusing on the well-identified effect of lagged sports, there is a decrease in the probability of moderate physical
health of $-2.5\%\,(p < 0.001)$, combined with a increase in the probability of very good physical health
of $2.6\%\,(p < 0.001)$. Interesting is the finding that the probabilities of reporting poor or excellent health
are only weakly influenced by instantaneous and lagged sports, indicating that at the extremes of physical health,
the effect of sports is overwhelmed by the effects of other (unmeasured) factors.
The total effect of sports on MHI5 mediated by physical health is found to be $x \pm y$. % TODO

Combining the direct effect and the effect through the mediators, we conclude that engaging in exercise \textit{increases/decreases}
MHI5 on average by \textit{TODO}, equivalent to a long-run effect of \textit{TODO * 4.55}.
% TODO compare to nonmediation

The largest effect size found in mediation regressions is $0.044$, which is quite small, justifying the approximation
of using linear models over a logit analysis.
% TODO discuss parallel mediation

\begin{table}[htbp]
    \centering
    \caption{Parameters of interest in the mediation analysis.
    Regressands are in bold.
    Standard errors in total effect are determined by the delta method}
    \label{tab:results:mediation_regression}
    \begin{tabular}{
        L
        S[table-format=2.3]
        S[table-format=2.3]
        S[table-format=2.3]
        S[table-format=1.3]
    }
    \toprule

    \textbf{Regressor} & \textbf{Estimate} & \textbf{Std. Error} & \textbf{z-value} & \textbf{p-value} \\

    \midrule

    \multicolumn{5}{l}{\textbf{MHI5}} \\
    MHI5$_{t-1}$                    & 0.369     & 0.007 & 51.366    & 0.000 *** \\
    MHI5$_{t-2}$                    & 0.228     & 0.008 & 29.190    & 0.000 *** \\
    MHI5$_{t-3}$                    & 0.183     & 0.007 & 24.908    & 0.000 *** \\

    sports$_t$                      & -0.023    & 0.169 & -0.133    & 0.894 \\
    sports$_{t-1}$                  & -0.303    & 0.173 & -1.751    & 0.080 $^+$ \\

    disease status$_t$              & -0.229    & 0.122 & -1.873    & 0.061 $^+$ \\

    \multicolumn{5}{l}{\textit{physical health} (excellent)} \\
    good                            & -4.437    & 0.243 & -18.256   & 0.000 *** \\
    moderate                        & -9.004    & 0.328 & -27.480   & 0.000 *** \\
    poor                            & -14.008   & 0.806 & -17.379   & 0.000 *** \\
    very good                       & -1.735    & 0.244 & -7.107    & 0.000 *** \\

    \midrule

    \multicolumn{5}{l}{\textbf{Disease status}} \\
    sports$_t$                      & 0.006     & 0.004 & 1.3181    & 0.188 \\
    sports$_{t-1}$                  & 0.007     & 0.005 & 1.478     & 0.139 \\

    \midrule

    \multicolumn{5}{l}{\textbf{Physical health} (excellent)} \\
    good                            & \multicolumn{4}{l}{} \\
    \hspace{3mm} sports$_t$         & -0.021    & 0.007 & -3.153    & 0.002 ** \\
    \hspace{3mm} sports$_{t-1}$     & -0.011    & 0.007 & -1.615    & 0.106 \\

    moderate                        & \multicolumn{4}{l}{} \\
    \hspace{3mm} sports$_t$         & -0.022    & 0.005 & -4.722    & 0.000 *** \\
    \hspace{3mm} sports$_{t-1}$     & -0.025    & 0.005 & -5.175    & 0.000 *** \\

    poor                            & \multicolumn{4}{l}{} \\
    \hspace{3mm} sports$_t$         & -0.007    & 0.002 & -4.824    & 0.000 *** \\
    \hspace{3mm} sports$_{t-1}$     & 0.001     & 0.002 & 0.599     & 0.549 \\

    very good                       & \multicolumn{4}{l}{} \\
    \hspace{3mm} sports$_t$         & 0.044     & 0.005 & 8.427     & 0.000 *** \\
    \hspace{3mm} sports$_{t-1}$     & 0.026     & 0.005 & 4.897     & 0.000 *** \\

    \midrule

    Direct effect                   & -0.303    & 0.173 & -1.751    & 0.080 $^+$ \\
    Effect physical health          & \multicolumn{4}{l}{} \\
    Effect disease status           & \multicolumn{4}{l}{} \\
    Total effect                    & \multicolumn{4}{l}{} \\

    \midrule

    Observations    & \multicolumn{4}{l}{12920} \\
    $\chi^2$        & \multicolumn{4}{l}{6547.9 ($df = 1956$, $p = 0.000$)} \\
    CFI             & \multicolumn{4}{l}{0.969 (cutoff = 0.95)} \\
    TLI             & \multicolumn{4}{l}{0.960 (cutoff = 0.95)} \\
    RMSEA           & \multicolumn{4}{l}{0.013 (cutoff = 0.06)} \\
    SRMR            & \multicolumn{4}{l}{0.039 (cutoff = 0.08)} \\

    \bottomrule

    \multicolumn{5}{l}{\textit{Significance levels:} $^+$ 0.10, * 0.05, ** 0.01, *** 0.001} \\
\end{tabular}
\end{table}

\section{Discussion and Future Research}
\label{sec:results:discussion}
% TODO: make good text go be good
There are two possible interpretations of the results. First, the null hypothesis that there is no long-term effect of
exercise on mental health cannot be rejected based on the data.
Alternatively, neither can the null hypothesis that observational data does not provide sufficient evidence to
answer the research question. Both interpretations will now be discussed.

In this sense, the present study is consistent with the outcomes
of randomised controlled trials in the literature, many of which also do not find evidence for a beneficial effect
(e.g. \citeA{chalder2012facilitated}).

This is evidence for the impact of the various data quality issues discussed, but is perhaps also indicative of the
complexities of mental wellbeing that make it hard to explain its variability in terms of the studied variables.
The fit indices represent good to excellent fit.

However, the data quality issues also lend credence to the idea that analysis of observational data will in general
not be powered for the present causal analysis, as it may be infeasible to justify the ignorability assumption
without an experimental design that explicitly controls for confounders. Mainly, the MNAR missingness and the issues resulting
from self-reported data in questionnaires cause a downward bias that compromise statistical power.

% TODO future research angles based on the interpretation
