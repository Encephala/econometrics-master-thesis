For the definitive models in \textit{lavaan} syntax, refer to the programming code (\cref{chap:app:code}).

\section{Excluding Mediation}
\label{sec:results:no_mediation}

To reiterate, in this approach mediation regressions are not included and neither are the mediators included as regressors,
gaining model parsimony at risk of violating the ignorability assumption.
The regression parameters are listed in \cref{tab:results:basic_regression}. Since besides the autoregressive
terms each variable included is a binary variable, the parameters can be interpreted as the percentage point change
in mental wellbeing (MHI5) as a result of that variable.
In general, very few estimated parameters are significant, and even those that are represent only a marginal impact,
no greater than $1.4$ points for an income above € 50\,000 as compared to no income. Contrast this with the interquartile
range in MHI5 of about $20$ (\cref{fig:data:sample_moments_y_x}).

% TODO: Discuss fit indices

The results may especially be put into question for the very small instantaneous association between sports and MHI5.
While statistically significant, an average improvement of $0.48$ on a scale of 0 to 100 for individuals who exercise
is in stark contrast with cross-sectional analyses in the literature. Recall for instance the 43\% decrease in days of poor mental
health found in \citeA{chekroud2018association}, or the review by \citeA{noetel2024effect} who find effect sizes of between
$0.2$ and $0.8$ times the sample standard deviation on various measures.
The estimated long-term effect of exercise is highly insignificant ($p = 0.888$), and hence the null hypothesis that
no such long-term benefit exists is maintained.

In time series analysis, it is customary to examine the stability of the data, as stability is typically a necessary
assumption for estimation. While SEM does not rely explicitly on this assumption, absence of stability would at least make the
assumption that parameters are constant in time contentious, and the derivation of the long-term effect as a change
in equilibrium value (\cref{eq:methods:long_run_effect}) does rely on stability.
Based on the reported autoregressive coefficients, the roots of the characteristic polynomial are $z_1 = 1.10$,
$z_2 = -1.16 + 1.78i$ and $z_3 = -1.16 - 1.78i$, which all fall decidedly outside of the unit circle,
indeed indicating a stable time series.

\newcolumntype{L}{>{\hspace*{3mm}}l}
\begin{table}[htbp]
    \centering
    \caption{Parameter estimates and fit indices for the base regression.
    Estimates are changes in mean MHI5-scores with respect to the dummy level in parentheses.
    Fit indices are robust variants where applicable}
    \label{tab:results:basic_regression}
    \begin{tabular}{
        L
        S[table-format=2.3] % three decimals, two digits before
        S[table-format=2.3]
        S[table-format=2.3]
        S[table-format=1.3]
    }
    \toprule

    \textbf{Regressor} & \textbf{Estimate} & \textbf{Std. Error} & \textbf{z-value} & \textbf{p-value} \\

    \midrule

    MHI5$_{t-1}$                    & 0.397     & 0.007 & 54.254    & 0.000 *** \\
    MHI5$_{t-2}$                    & 0.248     & 0.008 & 30.737    & 0.000 *** \\
    MHI5$_{t-3}$                    & 0.202     & 0.007 & 26.939    & 0.000 *** \\

    sports$_t$                      & 0.432     & 0.177 & 2.746     & 0.013 * \\
    sports$_{t-1}$                  & -0.025    & 0.177 & -0.140    & 0.888 \\

    \multicolumn{5}{l}{\textit{age} (below 18 years)} \\
    18-24 years                     & -0.343    & 0.372 & -0.923    & 0.356 \\
    25-39 years                     & -0.107    & 0.438 & -0.244    & 0.807 \\
    40-66 years                     & 0.700     & 0.443 & 1.579     & 0.114 \\
    over 67 years                   & 0.331     & 0.490 & 0.676     & 0.499 \\

    \multicolumn{5}{l}{\textit{income} (none)} \\
    below € 15\,000                 & 1.042     & 0.536 & 1.944     & 0.052 $^+$ \\
    over € 50\,000                  & 1.331     & 1.178 & 1.129     & 0.259 \\

    \multicolumn{5}{l}{\textit{immigration status} (Dutch)} \\
    first generation western        & -0.688    & 0.257 & -2.683    & 0.007 ** \\
    first generation non-western    & -0.959    & 0.268 & -3.576    & 0.000 *** \\
    second generation western       & -0.403    & 0.223 & -1.809    & 0.070 $^+$ \\
    second generation non-western   & -0.245    & 0.277 & -0.885    & 0.376 \\

    \multicolumn{5}{l}{\textit{gender} (female)} \\
    male                            & 0.562     & 0.103 & 5.474     & 0.000 *** \\

    \multicolumn{5}{l}{\textit{marital status} (divorced)} \\
    married                         & 0.155     & 0.178 & 0.873     & 0.383 \\
    never been married              & -0.651    & 0.207 & -3.139    & 0.002 ** \\
    separated                       & -0.813    & 0.859 & -0.947    & 0.344 \\
    widow or widower                & 0.337     & 0.289 & 1.167     & 0.243 \\

    \multicolumn{5}{l}{\textit{education level} (havo-vwo)} \\
    hbo                             & -0.176    & 0.193 & -0.914    & 0.361 \\
    mbo                             & 0.021     & 0.196 & 0.110     & 0.913 \\
    primary school                  & -0.382    & 0.273 & -1.401    & 0.161 \\
    vmbo                            & -0.056    & 0.205 & -0.273    & 0.785 \\
    university (wo)                 & -0.436    & 0.218 & -0.273    & 0.046 * \\

    \multicolumn{5}{l}{\textit{employment status} (employed)} \\
    homemaker                       & -0.510    & 0.195 & -2.618    & 0.009 ** \\
    retired                         & 0.063     & 0.191 & 0.327     & 0.744 \\
    student                         & -0.594    & 0.337 & -1.764    & 0.078 $^+$ \\
    unable to work                  & -1.409    & 0.313 & -4.505    & 0.000 *** \\
    unemployed                      & -0.415    & 0.365 & -1.137    & 0.256 \\

    \midrule

    Observations    & \multicolumn{4}{l}{12920} \\
    Parameters      & \multicolumn{4}{l}{225 (180 equality constraints)} \\
    $\chi^2$        & \multicolumn{4}{l}{1530.6 ($df = 270$, $p = 0.000$)} \\
    CFI             & \multicolumn{4}{l}{0.956 (cutoff = 0.95)} \\
    TLI             & \multicolumn{4}{l}{0.951 (cutoff = 0.95)} \\
    RMSEA           & \multicolumn{4}{l}{0.019 (cutoff = 0.06)} \\
    SRMR            & \multicolumn{4}{l}{0.026 (cutoff = 0.08)} \\

    \bottomrule

    \multicolumn{5}{l}{\textit{Significance levels:} $^+$ 0.10, * 0.05, ** 0.01, *** 0.001} \\
\end{tabular}
\end{table}

\section{Including Mediation}
\label{sec:results:mediation}

The main findings of interest in the mediation analysis are to what extent the effect is explained by the mediators,
and the degree to which the mediators confounded the result as indicated by the change in the estimated total effect.
The model contains an overwhelming number of parameters, which would be difficult to interpret.
Hence, \cref{tab:results:mediation_regression} lists the main parameters of interest, namely those of sports
in the mediation regressions and those of sports and the mediators in the main regression.
For the complete parameter estimates, refer to \cref{chap:app:mediation_regression}.
\Cref{tab:results:mediation_total_effect} reports the total effects derived from the results.

% TODO update with numbers with lagged mediators-numbers!
The instantaneous effect of sports as well as the (direct) lagged effect are now notably more negative than when no mediators
are included, changing by $-0.390 \pm 0.24$ and $-0.142 \pm 0.25$ respectively. This points towards there indeed being a
positive effect through the mediators, with the large estimated effects of physical health showing it especially is an important
factor. The change in the instantaneous association also provides some evidence for mediation of the reverse
action, i.e. that being sick or being in poor physical health decreases chance to exercise, although the difference
is not significant ($p = 0.110$ based on the above numbers).
The effect of lagged sports is insignificant at $-0.167 \pm 0.172$ with a p-value of $0.332$.

Being ill decreases mental wellbeing ($p < 0.001$), but only moderately with a $-1.384$ percentage point change
in MHI5 on average.
Interesting is that being being ill last year has a similarly large but positive effect of $1.474$ percentage points,
indicating that having recovered from a disease improves wellbeing and/or that having just fallen ill is associated
with a larger mental burden than having been ill for a longer time.
We do not find a significant long-term effect of exercise on disease status ($p = 0.155$).
This indicates that exercise does not seem to strengthen the immune system in the long run,
and because the LISS panel also queries bone fractures, exercise may also not have a beneficial effect towards
strengthening the skeleton, regardless of for instance \citeA{hong2018effects}'s findings that exercise
increases bone mass.
However, because this includes various forms of disease simultaneously, the finding does not preclude a positive effect of exercise on
some forms of disease as in \citeA{westcott2012resistance}, if those effects are offset with a negative effect on other forms
of disease.
In total, the causal long-term effect of physical exercise on mental wellbeing as moderated by disease is $-0.009$
percentage points, which is statistically insignificant at $p = 0.168$.

Physical health is found to have by far the greatest impact on MHI5 of any regressor,
with effect sizes as high as $-19.840$ percentage points.
There is a clear trend that worse physical health is associated with worse mental health.
Strikingly however, the lagged effects all have an opposite sign, that is, a positive association with mental wellbeing.
This is evidence of a similar rebound mechanism as with disease, where having recently had physical health improve is
cause for happiness, and vice versa.
Exercise generally relates to better physical health, but effect sizes are modest, with the greatest change found
being that exercise is associated with a 4.0\% increased probability of being in good physical health.
Lagged exercise significantly decreases probability of moderate physical health ($-2.4\%$, $p < 0.001$) while increasing
the probability of very good mental health ($2.3\%$, $p < 0.001$).
Interesting is the finding that the probabilities of reporting poor or excellent health
are only weakly influenced by instantaneous and lagged sports, indicating that at the extremes of physical health,
the effects of sports are overwhelmed by the effects of other (unmeasured) factors.
Combined with the large detrimental effect of last year's physical health on MHI5, we find a causal effect
of $-0.117 \pm 0.022$ percentage points MHI5 ($p < 0.001$). The negative sign of this effect is contrary to expectations based on the
literature, which find a positive effect if any. This oddity is due to the aforementioned rebound effect.

MHI5 is found to be positively predicted instantaneously by being overweight ($1.241$, $p = 0.025$) and
being obese ($1.232$, $p = 0.041$), both as compared to being underweight. These associations are actually more positive
than the effect of being normal weight, though not statistically significantly so.
Because we are controlling for physical health and disease status, these findings suggest that the effects of a negative
body image are more significant in underweight individuals than in overweight or obese individuals.
There is a significant instantaneous association between sports and being normal weight ($1.7\%$, $p = 0.009$)
as well as being obese ($-1.8\%$, $p < 0.001$), with no association found for being underweight or overweight.
Lagged exercise only has a significant effect on the probability of being normal weight, increasing it by $0.9\%$ ($p = 0.038$),
with all other effects trending towards decreased probabilities.
This small change leads to a total effect of sports on MHI5 as mediated by BMI of $0.002$ with an associated p-value of $0.095$.

Combining the direct effect and the effects through the three mediators, we conclude that engaging in exercise decreases
MHI5 on average by $-0.290 \pm 0.174$ percentage points ($p = 0.095$), equivalent to a long-run effect of $-1.45$
percentage points.
The total effect is but weakly significant because the effect through physical health is overshadowed by the uncertainty
in the direct effect.
This is a small impact as compared to the IQR of $20$ percentage points, and considering the weak significance
the negative sign should not be given much interpretation beyond the aforementioned rebound effect in physical health.
The total effect is not very different from the direct effect of $-0.025$ found in in the analysis without mediation
(\cref{tab:results:basic_regression}), from which we may conclude that the mediators did not appear to have a significant
effect as unmeasured confounders.

% quality of results:
% TODO fit indices
% TODO largest effect size in all the mediation regressions, not just in the variables of interest?
% or justify that the approximation is most relevant among the variables of interest
Out of the parameters of interest, the largest effect size found in mediation regressions is $0.044$.
Some other parameters are larger, like $0.283$ for the effect of being over 67 years old on disease status,
but none of the parameters of interest are as large and the vast majority of parameters is below $0.050$,
so the approximation of using linear models over a logit analysis is largely justified.
% TODO discuss parallel mediation

\begin{table}[htbp]
    \scriptsize
    \centering
    \caption{Parameters of interest in the mediation analysis; controls are not reported.
    Regressands are in bold.
    Standard errors in total effect are determined by the delta method}
    \label{tab:results:mediation_regression}
    \begin{tabular}{
        L
        S[table-format=2.3]
        S[table-format=2.3]
        S[table-format=2.3]
        S[table-format=1.3]
    }
    \toprule

    \textbf{Regressor} & \textbf{Estimate} & \textbf{Std. Error} & \textbf{z-value} & \textbf{p-value} \\

    \midrule

    \multicolumn{5}{l}{\textbf{MHI5$_t$}} \\
    MHI5$_{t-1}$                    & 0.391     & 0.007 & 54.301    & 0.000 *** \\
    MHI5$_{t-2}$                    & 0.227     & 0.008 & 29.093    & 0.000 *** \\
    MHI5$_{t-3}$                    & 0.182     & 0.007 & 25.401    & 0.000 *** \\

    sports$_t$                      & 0.042     & 0.168 & 0.249     & 0.803 \\
    sports$_{t-1}$                  & -0.167    & 0.172 & -0.969    & 0.332 \\

    disease status$_t$              & -1.384    & 0.216 & -6.398    & 0.000 *** \\
    disease status$_{t-1}$          & 1.474     & 0.221 & 6.677     & 0.000 *** \\

    \multicolumn{5}{l}{\textit{physical health}$_t$ (excellent)} \\
    good                            & -6.366    & 0.310 & -20.566   & 0.000 *** \\
    moderate                        & -12.818   & 0.402 & -31.907   & 0.000 *** \\
    poor                            & -19.840   & 0.861 & -23.050   & 0.000 *** \\
    very good                       & -2.785    & 0.283 & -9.859    & 0.000 *** \\
    \multicolumn{5}{l}{\textit{physical health}$_{t-1}$ (excellent)} \\
    good                            & 2.952     & 0.342 & 8.631     & 0.000 *** \\
    moderate                        & 6.176     & 0.422 & 14.642    & 0.000 *** \\
    poor                            & 10.746    & 0.820 & 13.102    & 0.000 *** \\
    very good                       & 1.694     & 0.320 & 5.288     & 0.000 *** \\

    \multicolumn{5}{l}{\textit{BMI}$_t$ (underweight)} \\
    normal weight                   & 0.306     & 0.518 & 0.591     & 0.554 \\
    overweight                      & 1.241     & 0.552 & 2.247     & 0.025 * \\
    obese                           & 1.232     & 0.603 & 2.042     & 0.041 * \\
    \multicolumn{5}{l}{\textit{BMI}$_{t-1}$ (underweight)} \\
    normal weight                   & 0.347     & 0.567 & 0.612     & 0.541 \\
    overweight                      & -0.020    & 0.602 & -0.033    & 0.973 \\
    obese                           & 0.207     & 0.656 & 0.315     & 0.752 \\

    \midrule

    \multicolumn{5}{l}{\textbf{Disease status$_t$}} \\
    sports$_t$                      & -0.007    & 0.004 & -1.608    & 0.108 \\
    sports$_{t-1}$                  & -0.006    & 0.004 & -1.423    & 0.155 \\

    \midrule

    \multicolumn{5}{l}{\textbf{Physical health$_t$} (excellent)} \\
    good                            & \multicolumn{4}{l}{} \\
    \hspace{3mm} sports$_t$         & -0.009    & 0.007 & -1.400    & 0.162 \\
    \hspace{3mm} sports$_{t-1}$     & -0.009    & 0.007 & -1.400    & 0.162 \\

    moderate                        & \multicolumn{4}{l}{} \\
    \hspace{3mm} sports$_t$         & -0.020    & 0.007 & -3.007    & 0.003 ** \\
    \hspace{3mm} sports$_{t-1}$     & -0.024    & 0.005 & -4.906    & 0.000 *** \\

    poor                            & \multicolumn{4}{l}{} \\
    \hspace{3mm} sports$_t$         & -0.007    & 0.002 & -4.610    & 0.000 *** \\
    \hspace{3mm} sports$_{t-1}$     & 0.001     & 0.002 & 0.957     & 0.339 \\

    very good                       & \multicolumn{4}{l}{} \\
    \hspace{3mm} sports$_t$         & 0.040     & 0.005 & 7.810     & 0.000 *** \\
    \hspace{3mm} sports$_{t-1}$     & 0.023     & 0.005 & 4.422     & 0.000 *** \\

    \midrule

    \multicolumn{5}{l}{\textbf{BMI$_t$} (underweight)} \\
    normal weight                   & \multicolumn{4}{l}{} \\
    \hspace{3mm} sports$_t$         & 0.017     & 0.004 & 4.011     & 0.009 ** \\
    \hspace{3mm} sports$_{t-1}$     & 0.009     & 0.005 & 2.075     & 0.038 * \\

    overweight                      & \multicolumn{4}{l}{} \\
    \hspace{3mm} sports$_t$         & 0.003     & 0.005 & 0.632     & 0.527 \\
    \hspace{3mm} sports$_{t-1}$     & -0.002    & 0.005 & -0.329    & 0.742 \\

    obese                           & \multicolumn{4}{l}{} \\
    \hspace{3mm} sports$_t$         & -0.018    & 0.003 & -5.894    & 0.000 *** \\
    \hspace{3mm} sports$_{t-1}$     & -0.004    & 0.003 & -1.372    & 0.170 \\

    \midrule

    Observations    & \multicolumn{4}{l}{12920} \\
    Parameters      & \multicolumn{4}{l}{3100 (1891 equality constraints)} \\
    $\chi^2$        & \multicolumn{4}{l}{52855.2 ($df = 5061$, $p = 0.000$)} \\
    CFI             & \multicolumn{4}{l}{0.864 (cutoff = 0.95)} \\
    TLI             & \multicolumn{4}{l}{0.835 (cutoff = 0.95)} \\
    RMSEA           & \multicolumn{4}{l}{0.027 (cutoff = 0.06)} \\
    SRMR            & \multicolumn{4}{l}{0.089 (cutoff = 0.08)} \\

    \bottomrule

    \multicolumn{5}{l}{\textit{Significance levels:} $^+$ 0.10, * 0.05, ** 0.01, *** 0.001} \\
    \end{tabular}
\end{table}

\begin{table}
    \centering
    \caption{Effect through each mediator and direct effect, as derived from \cref{tab:results:mediation_regression}}
    \label{tab:results:mediation_total_effect}
    \begin{tabular}{
        L
        S[table-format=2.3]
        S[table-format=2.3]
        S[table-format=2.3]
        S[table-format=1.3]
    }

    \toprule

    \textbf{Regressor} & \textbf{Estimate} & \textbf{Std. Error} & \textbf{z-value} & \textbf{p-value} \\

    \midrule

    Direct effect                   & -0.167    & 0.172 & -0.969    & 0.332 \\
    Effect disease status           & -0.009    & 0.007 & -1.380    & 0.168 \\
    Effect physical health          & -0.117    & 0.022 & -5.399    & 0.000 *** \\
    Effect BMI                      & 0.002     & 0.003 & 0.762     & 0.761 \\
    Total effect                    & -0.290    & 0.174 & -1.671    & 0.095 \\

    \bottomrule

    \multicolumn{5}{l}{\textit{Significance levels:} $^+$ 0.10, * 0.05, ** 0.01, *** 0.001} \\
    \end{tabular}
\end{table}

\section{Discussion and Future Research}
\label{sec:results:discussion}
% TODO: make good text go be good

There are two main conclusions that can be drawn from the results. First, we do not find strong evidence for a
long-term benefit of exercise on mental health. Second, the complex model design obscures many of the effects of
exercise that are very much of real-world interest.
Both conclusions inform import decisions in future research and will thus be discussed in the following.

The absence of a strong long-term effect means that the associated mechanisms through which exercise may have a lasting
effect on mental wellbeing are weak. These include structural changes to the brain, physiological adaptations to for instance
improve body image and improve sleep, and better habits through improved self-efficacy.
Improvements in these factors due to exercise are either small, or they do not persist when one stops exercising.
However, there is an extensive literature on the persistency of exercise's benefits on cognition, referred to as
cognitive reserve \cite{cheng2016cognitive,stern2009cognitive}. This makes the prior interpretation favoured.
While a lack of effect finding is consistent with some literature which find no strong benefit (e.g. \citeA{chalder2012facilitated}),
it stands in stark contrast with other research (e.g. \citeA{philippot2022impact}) and even with this extensive literature
on cognitive reserve.
The present finding of an insignificant effect, in fact one that is negative if anything, may be partially attributable
to the biases resulting from self-reporting and MNAR data.
Varying impacts of this bias have been found in the literature, from a negligible impact in \citeA{leroux2012role}
to effects frequently switching from significantly positive to significantly negative and vice versa in \citeA{brown2018mental}.
This makes it hard to say how relevant it is.
Beyond these biases, a more appropriate set of mediators than the small presently used set of three may help
in finding an effect. For instance, body image may be measured directly as opposed to BMI for better precision,
and especially the very broad concept of "physical health" may be refined.
There is thus a need for further research in a similar vein to the present study but with more precise ways of measuring
relevant mediators and outcomes, as well as with close attention paid to avoiding missingness.

In pursuit of a causal effect, this study only examines the effect of exercise a year ago on present mental health,
or even of exercise two years ago for the mediated effects.
This precludes estimating the impact of many of the hypothesised mechanisms, like the endorphin release which
likely only lasts on the order of hours, or the improved self-efficacy which may well only persist on the order of
weeks to months after one stops exercising.
Additionally, if the benefit in for instance self-efficacy is instantaneous but not greater for an individual
who has been exercising for multiple years as compared to someone who just started, the effect cannot be measured.
In general, any mechanism whose effect is not both cumulative with the total amount of exercise and (approximately)
permanent cannot be quantified in this approach, yet these mechanisms are manifold.
Considering the literature either reports an insignificant or a positive effect, yet the present finding is a
negative effect through physical health and a weakly negative total effect, it is likely that these
mechanisms are important to consider.
In fact, because the instantaneous associations in \cref{tab:results:mediation_regression} are consistently larger
than the lagged effects and point consistently towards a positive association,
the present data even leaves much room for these effects.
This informs an emphasis in future research on the short-term or direct effect of exercise over the long-term effect.

In summary then, further research into the benefit of exercise for mental wellbeing should emphasise data quality
while finding an alternative approach to control for reverse causality.
For the purpose of data quality, it is important to conduct a prospective study, where the variables being measured
are predetermined for the specific purpose of the study, as opposed to a retrospective study on a general dataset.
That is, variables should be selected to more accurately dissect the suspected mediators based on expert knowledge
of the mechanisms at play. Furthermore, as opposed to aggregating everything to a singular MHI5 score,
the impact of sports on various components of mental wellbeing may be examined separately, as for instance
\citeA{atlantis2004effective} find appreciably different outcomes on such different components.
There is an additional benefit to be had in using an objective measure of mental health over self-reporting and in
tight quality control so as to avoid MNAR missingness, although perhaps solving those problems is wishful thinking.
The randomisation in RCTs would control effectively for reverse causality. That being said, it is surprising that
the findings even among RCTs in the literature is so varied, and the present study yields no insight as to why that may be.
A further consideration is that tightly controlling the environment when measuring long-term effects is infeasible,
necessarily making measuring this effect more complicated, although a difference-in-differences design may partially
offset this downside.
In an observational setting it is much more feasible to measure long-term effects, but the present study shows
that causal analysis in such a context is infeasible. An alternative approach may be to use instrumental variables,
but considering the complexity of all the psychological, physiological and social mechanisms at play in the subject of
mental health, it is not likely that an instrument can be found for which the assumption that it is exogenous
with mental health is justifiable.
It appears more worthwhile for observational studies to be focused on Granger causality, which combined with
mechanistic research and randomised controlled trials may provide a better means to discover causality.
