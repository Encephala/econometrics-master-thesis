For the definitive models in \textit{lavaan} syntax, refer to the programming code (\cref{chap:app:code}).

\section{Excluding Mediation}
\label{sec:results:no_mediation}

To reiterate, in this approach mediation regressions are not included and neither are the mediators included as regressors,
gaining model parsimony at risk of violating the ignorability assumption.
The regression parameters are listed in \cref{tab:results:basic_regression}. Since besides the autoregressive
terms each variable included is a binary variable, the parameters can be interpreted as the percentage point change
in mental wellbeing (MHI5) as a result of that variable.
In general, very few estimated parameters are significant, and even those that are represent only a marginal impact,
no greater than $1.4$ points for an income above € 50\,000 as compared to no income. Compare this to the interquartile
range in MHI5 of about $20$ (\cref{fig:data:sample_moments_y_x}).
This is evidence for the impact of the various data quality issues discussed, but is perhaps also indicative of the
complexities of mental wellbeing that make it hard to explain its variability in terms of the studied variables.
The fit indices represent good to excellent fit.

The results may especially be put into question for the very small instantaneous association between sports and MHI5.
While statistically significant, an average improvement of $0.48$ on a scale of 0 to 100 for individuals who exercise
is in stark contrast with cross-sectional analyses in the literature. Recall for instance the 43\% decrease in days of poor mental
health found in \citeA{chekroud2018association}, or the review by \citeA{noetel2024effect} who find effect sizes of between
$0.2$ and $0.8$ times the sample standard deviation on various measures.
The estimated long-term effect of exercise is highly insignificant ($p = 0.786$), and hence the null hypothesis that
no such long-term benefit exists is maintained. In this sense, the present study is consistent with the outcomes
of randomised controlled trials in the literature, many of which also do not find evidence for a beneficial effect
(e.g. \citeA{chalder2012facilitated}).
However, the data quality issues also lend credence to the idea that analysis of observational data will in general
not be powered for the present causal analysis, as it may be infeasible to justify the ignorability assumption
without an experimental design that explicitly controls for confounders. Mainly, the MNAR missingness and the issues resulting
from self-reported data in questionnaires cause a downward bias that compromise statistical power.

In time series analysis, it is customary to examine the stability of the data, as stability is typically a necessary
assumption for estimation. While SEM does not rely explicitly on this assumption, absence of stability would make the
assumption that parameters are constant in time contentious. Based on the reported autoregressive coefficients,
the roots of the characteristic polynomial are $z_1 = 1.10$, $z_2 = -1.16 + 1.78i$ and $z_3 = -1.16 - 1.78i$, which all
fall well outside of the unit circle, indicating a stable time series.

\newcolumntype{L}{>{\hspace*{3mm}}l}
\begin{table}[htbp]
    \centering
    \caption{Parameter estimates and fit indices for the base regression.
    Estimates are changes in mean MHI5-scores with respect to the dummy level in parentheses.
    Fit indices are robust variants where applicable}
    \label{tab:results:basic_regression}
    \begin{tabular}{
        L
        S[table-format=2.3] % three decimals, two digits before
        S[table-format=2.3]
        S[table-format=1.2]
        S[table-format=1.3]
    }
    \toprule

    \textbf{Regressor} & \textbf{Estimate} & \textbf{Std. Error} & \textbf{z-value} & \textbf{p-value} \\

    \midrule

    MHI5$_{t-1}$                    & 0.396     & 0.007 & 54.276    & 0.000 *** \\
    MHI5$_{t-2}$                    & 0.247     & 0.008 & 30.717    & 0.000 *** \\
    MHI5$_{t-3}$                    & 0.202     & 0.007 & 26.988    & 0.000 *** \\

    sports$_t$                      & 0.482     & 0.176 & 2.744     & 0.006 ** \\
    sports$_{t-1}$                  & 0.048     & 0.177 & 0.272     & 0.786 \\

    \multicolumn{5}{l}{\textit{age} (below 18 years)} \\
    18-24 years                     & -0.307    & 0.372 & -0.827    & 0.408 \\
    25-39 years                     & -0.053    & 0.438 & -0.121    & 0.903 \\
    40-66 years                     & 0.765     & 0.444 & 1.724     & 0.085 $^+$ \\
    over 67 years                   & 0.404     & 0.491 & 0.824     & 0.410 \\

    \multicolumn{5}{l}{\textit{income} (none)} \\
    below € 15\,000                 & 1.007     & 0.535 & 1.882     & 0.060 $^+$ \\
    over € 50\,000                  & 1.372     & 1.178 & 1.165     & 0.244 \\

    \multicolumn{5}{l}{\textit{immigration status} (Dutch)} \\
    first generation western        & -0.684    & 0.256 & -2.667    & 0.008 ** \\
    first generation non-western    & -0.946    & 0.268 & -3.535    & 0.000 *** \\
    second generation western       & -0.423    & 0.233 & -1.897    & 0.058 $^+$ \\
    second generation non-western   & -0.215    & 0.277 & -0.776    & 0.438 \\

    \multicolumn{5}{l}{\textit{gender} (female)} \\
    male                            & 0.571     & 0.103 & 5.560     & 0.000 *** \\

    \multicolumn{5}{l}{\textit{marital status} (divorced)} \\
    married                         & 0.150     & 0.178 & 0.846     & 0.397 \\
    never been married              & -0.655    & 0.207 & -3.158    & 0.002 ** \\
    separated                       & -0.816    & 0.866 & -0.942    & 0.346 \\
    widow or widower                & 0.331     & 0.288 & 1.147     & 0.251 \\

    \multicolumn{5}{l}{\textit{education level} (havo-vwo)} \\
    hbo                             & -0.212    & 0.193 & -1.098    & 0.272 \\
    mbo                             & 0.019     & 0.196 & 0.099     & 0.921 \\
    primary school                  & -0.346    & 0.273 & -1.268    & 0.205 \\
    vmbo                            & -0.030    & 0.206 & -0.146    & 0.884 \\
    university (wo)                 & -0.483    & 0.218 & -2.214    & 0.027 $^+$ \\

    \multicolumn{5}{l}{\textit{employment status} (employed)} \\
    homemaker                       & -0.496    & 0.195 & -2.548    & 0.011 * \\
    retired                         & 0.106     & 0.192 & 0.554     & 0.580 \\
    student                         & -0.620    & 0.337 & -1.842    & 0.065 $^+$ \\
    unable to work                  & -1.354    & 0.312 & -4.344    & 0.000 *** \\
    unemployed                      & -0.390    & 0.363 & -1.074    & 0.283 \\

    \midrule

    Observations    & \multicolumn{4}{l}{12920} \\
    $\chi^2$        & \multicolumn{4}{l}{3106.4 ($df = 527$, $p = 0.000$)} \\
    CFI             & \multicolumn{4}{l}{0.953 (cutoff = 0.95)} \\
    TLI             & \multicolumn{4}{l}{0.945 (cutoff = 0.95)} \\
    RMSEA           & \multicolumn{4}{l}{0.019 (cutoff = 0.06)} \\
    SRMR            & \multicolumn{4}{l}{0.044 (cutoff = 0.08)} \\

    \bottomrule

    \multicolumn{5}{l}{\textit{Significance levels:} $^+$ $p < 0.10$, * $p < 0.05$, ** $p < 0.01$, *** $p < 0.001$} \\
\end{tabular}
\end{table}

\section{Including Mediation}
\label{sec:results:mediation}

The lack of evidence for a long-term effect may be attributable to confounding due to the mediators that are not included,
while they are very likely to play a role. Hence, to lend more credence to the findings, a proper mediation analysis
should also be done.
The analysis contains an overwhelming number of parameters, which would be difficult to interpret.
Hence, \cref{tab:results:mediation_regression} lists the main parameters of interest, namely those of sports
in the mediation regressions and those of sports and the mediators in the main regression.
For the complete parameter estimates, refer to \cref{chap:app:mediation_regression}.

\begin{table}[htbp]
    \caption{Epic stuff}
    \label{tab:results:mediation_regression}
\end{table}
