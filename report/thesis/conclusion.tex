For the purpose of answering the research question of whether physical exercise is an effective intervention
to improve mental wellbeing, the present study emphasises causal analysis in an observational setting, noting
the compelling mechanisms for reverse causality in this context.
The studied data is of the LISS panel, a representative Dutch panel with 7500 participants.
To be precise, the effect of previous exercise on present wellbeing is studied through two regression analyses,
one without modelling mediators and the other an explicit mediation analysis.
The prior finds a highly insignificant effect of $-0.025$ percentage points on the MHI5 scale ($p = 0.888$).
The latter finds an average effect of $-0.290$ percentage points, which is also not significant at $p = 0.095$.
The negative sign can be attributed partly to the statistical insignificance and partly to a significant rebound effect
that was found around physical health, where a poor physical condition
in the previous year was associated with better mental wellbeing, due to either happiness from recovery or lesser
sadness due to for instance developed coping mechanisms.

These findings mainly highlight the difficulty of causal analysis through observational data for a topic as complex
and nuanced as human mental wellbeing. Nevertheless, some directions for future research can be drawn from the present study.
Beyond the obvious but perhaps infeasible objective of gathering high-quality data,
there is a clear benefit in a prospective study design where the gathered data is tailored to the study objective,
as opposed to studying data generally available in public datasets.
The findings also underline once more the importance of randomisation in trials as a mechanism to control for potential
reverse causality, as opposed to post-factum analytical approaches.
For the purpose of studying causality in mental wellbeing, randomised controlled trials and mechanistic research are
likely more feasible than observational research, which appears a futile effort based on the present study.
