\section{Causal analysis}
\label{sec:methods:causal}

\subsection{Potential Outcome Framework}

In order to credibly make the claim that the estimated effect is a causal one, some a priori considerations are in order.
To this end, the review by \citeA{imbens2024causal} is considered.

Firstly, there is the stable unit treatment value assumption (SUTVA), which is the assumption that the outcome for one
individual is not influenced by treatment exposure (in this case, engagement in exercise) of another.
Since the subjects of the LISS panel are selected to be representative of the Netherlands as a whole, the subjects
live all over the country and are not likely to know each other, let alone influence each other. However, it is typical
for multiple individuals to partake within a single household. On average, it is found that about 1.6 individuals participate
per household. Within one household, we cannot rule out that individuals influence each other. In fact, \citeA{maltby2012contextual}
find that one's perception of the benefit of exercise is influenced by social norms, clearly violating the SUTVA.
However, because these violations are local to small groups in the dataset, the assumption is taken to hold at large for
the studied data.

Next, we formalise the effect of interest, namely the causal effect of exercise on mental wellbeing, using the
potential outcome.
For each individual $i$, $Y_i(T)$ is the potential outcome if they engage in exercise (treatment),
and $Y_i(C)$ is the potential outcome if they don't (control). The variable of interest then is the sample average
treatment effect
\begin{equation}
\label{eq:methods:sate}
    \tau_{\text{sample}} = \sum_{i=1}^N (Y_i(T) - Y_i(C)),
\end{equation}
which thanks to the representativeness of the sample is assumed to estimate the population average treatement.
This is where SUTVA is crucial, as without it the potential outcomes are not simply a function of individual $i$'s treatment,
but also of individual $j$'s treatment (where $j \neq i$), and so we could not meaningfully interpret the average
treatment effect calculated as in \cref{eq:methods:sate}.
For any one individual the treatment effect may deviate from the average effect, in the sense that it may be moderated by
for instance personal circumstances or genetics, but such analysis is left to further research.

Denote $X_i$ as the treatment assignment for individual $i$, $X_i \in \{C, T\}$. The crucial assumption is that the
\textit{potential} outcomes are not influenced by the treatment assignment, or formally, given some set of controls
$W_i$,
\begin{equation}
    (Y_i(T), Y_i(C)) \perp\!\!\!\perp X_i \mid W_i.
\end{equation}
This assumption is called the ignorability assumption or the assumption of no unobserved confounders.
The controls are included to lend credibility to the assumption, as they remove biases in comparing treated individuals
to individuals in the control group.
Selection bias, MNAR data and the assumption's namesake in no unobserved confounders can all equivalently be considered
violations of the ignorability assumption.
The crucial difficulty that necessitates the assumption is that for any one individual $i$,
only either $Y_i(T)$ or $Y_i(C)$ is observed, and thus we cannot directly observe the difference in potential outcomes.
However, under the ignorability assumption, this difference can be estimated by $Y_i(T) - Y_j(C)$ for $j \neq i$,
which is a quantity that can be observed directly.

\subsection{Reverse Causality}
Now consider the aim of the present study, namely estimating the effect of exercise on mental health, $Y_i(T) - Y_i(C)$.
If mental health influences the probability to engage in exercise, the ignorability assumption is violated.
If for instance better mental health makes one more likely to engage in exercise, and assuming the greater mental health
does not disappear as soon as one engages in exercise, then the \textit{potential} outcomes $Y_i$ are increased
for individuals with better mental health. That is, in the observed sample, $Y_i(T) - Y_j(C)$ will be inflated with respect to
the average treatment effect.
This holds true even if the treatment effect itself is not influenced by a priori mental health.

% Probably make this a subfigure with left CS and right panel
\begin{figure}[htbp]
    \centering
    \caption{Directed graph representation of reverse causality between MHI5-score and sports}
    \label{fig:methods:reverse_causality_crosssectional}
    \begin{tikzpicture}[
        node distance=0.8cm,
        main node/.style={circle,draw,font=\small\sffamily\bfseries,minimum size=1.8cm},
        every edge/.style={draw,thick,->}
    ]
        \node[main node] (mhi5) {MHI5};
        \node[main node, below=of mhi5] (sports) {SPORTS};
        \node[main node] (x) at ($(mhi5)!0.5!(sports) + (-3.5cm,0)$) {X};

        \path[bend left] (mhi5) edge (sports);
        \path[bend left] (sports) edge (mhi5);
        \path (x) edge (mhi5);
        \path (x) edge (sports);
    \end{tikzpicture}
\end{figure}

\section{Structural Equation Modelling}
\label{sec:methods:sem}
