\section{Regressors}
\label{sec:modelling:regressors}
As covariates, we include a basic set of sociodemographic variables available in the Background Variables of the LISS panel,
namely age, ethnicity, gender, marital status, education level, employment status and net household income.
Of these, the last four might be hypothesised as mediators based on the literature
(\citeA{spiker2014mental,hjorth2016mental,frijters2014effect,thomson2022income} respectively).
However, all of these variables were found to vary so little in time that including them as time-variant covariates caused
numerical instability, so they could only be included as time-invariants. This precludes modelling them as mediators
of the time-variant sports,
but the fact that they vary so little indicates that their effect as mediators could only be weak ($\zeta$ must be small),
so the error in estimating the total effect due to not modelling them as mediators could only be marginal.
Beyond this set, based on the literature and availability in the data, we include (self-reported) physical health
and presence of diagnosed diseases as additional mediators \cite{westcott2012resistance}.
While the Health study also queries more nuanced variables like hospital admittance and medication use, these are
highly correlated with general health and disease status (e.g. $\rho \approx 0.55$ for medication use and disease status),
and are thus excluded for the sake of parsimony, noting the significant increase in complexity that mediators bring.

Because only the long-term effect of sports on mental health is considered (recall \cref{sec:methods:reverse_causality}),
it would be natural to use the cumulative years that someone has exercised as the variable of interest, rather than the
yearly exercise status. However, because it is not known how many years one has exercised at the time of joining this panel,
this variable can only be roughly approximated through a running sum. Additionally, information missing in one year would
complicate using the variable in all future years.
As such, sports is used directly, while examining different orders of the distributed lags to model effects beyond
one year ahead.

This study chooses to not include individual-specific effects $\alpha_i$ in the model. This is because while they help justify
the ignorability assumption through eliminating unmeasured confounders, they also potentially obscure the effects of time-invariant mediators,
which equally invalidate the findings. The approach is thus instead to model individual-specific variability through the
time-invariant controls $z_i$. Additionally, inclusion of fixed effects entails estimating $N$ additional parameters
(the incidental parameter problem). Since it is likely that $N > \frac{p(p + 1)}{2}$, the model would not be identified.
That being said, nothing prevents modelling random effects. In fact, this is commonly done in
SEM, see for instance \citeA{heck2001multilevel}. However, the author leaves it as an avenue for further research.

As a minor final comment, some dummy levels of controls were excluded, because they occurred so infrequently in the data
that including them as regressors led to numerical instability.
These were a gender of ``other'' and, perhaps surprisingly, a household income between 15\,000 and 50\,000.

\section{Lag Selection}
\label{sec:modelling:lags}
For determining the autoregressive lag order $L_y$ and the distributed lag order $L_x$, cross validation was run as
described in \cref{sec:methods:cv}.
However, for computational tractability, mediators were not modelled explicitly as such but simply included as controls.
This should have only a minor impact on forecasting MHI5, but makes a higher number of repeats $R = 50$ feasible.
It is assumed that the optimal model structure found in this way also applies when mediation regressions are also
included.
Additionally, it was found that only applying FIML to the endogenous variables yielded very similar parameter estimates
and thus similar forecasts. To then further improve computational tractability in cross validation, listwise deletion
was used for the exogenous variables (i.e. the controls).

Both $L_y$ and $L_x$ were varied from 1 to the highest number available based on the data.
Due to the missing Health study in 2014, this is $L_y = 8$, and because the sports has only been included since 2012,
$L_x = 11$.
The optimal $L_y$ is first estimated, as it has a more profound role in the model than the distributed lag.
When doing so, $L_x = 1$ is used as the most minimal model. When optimising $L_x$, the optimal set of AR lags is used.
It could technically be possible that a better model fit is found when not including all lags between the lowest
and the highest lag, for instance having the set of lags $\{1, 2, 4\}$. This was however not tested as it seems improbable
given the mechanisms at play, and thus would likely be overfitting.
\Cref{fig:modelling:cv_lags} visualises the found RMSPEs.

\begin{figure}[htbp]
    \centering
    \caption{Cross validated out-of-sample forecasting accuracy for varying lag orders, measured by root mean square error.
    Errorbars are standard deviations of accuracy across folds. The horizontal line represents the 1-$\sigma$ decision rule.
    The smaller errorbars represent uncertainty in $\mu_{\text{RMSPE}} + \sigma_{\text{RMSPE}}$}
    \label{fig:modelling:cv_lags}
    \begin{subfigure}[t]{0.49\textwidth}
        \centering
        \includesvg[width=\textwidth]{figures/modelling/cv_AR.svg}
        \vspace{0.1em}
        \subcaption{$\mu_{\text{RMSPE}}$ for varying maximum autoregressive lag $L_y$}
    \end{subfigure}
    \hfill
    \begin{subfigure}[t]{0.49\textwidth}
        \centering
        \includesvg[width=\textwidth]{figures/modelling/cv_DL.svg}
        \vspace{0.1em}
        \subcaption{$\mu_{\text{RMSPE}}$ for varying maximum distributed lag $L_x$}
    \end{subfigure}
\end{figure}

For the autoregressive lags, the most complex model with AR lag order $L_y = 8$ performs best. The decision rule
then indicates $L_y = 3$ is best, though the estimation uncertainty also allows for choosing $L_y = 4$.
Recalling that the cross validation findings may not generalise as it is only done on the data for 2023, and noting that
simpler models generalise better, $L_y = 3$ is chosen.
For the distributed lags, there is no appreciable difference in forecasting error across the models, indicating that
the regression coefficients for higher order lags of sports are near zero.
The simplest model with $L_x = 1$ is clearly chosen.

\section{Fixing Parameters across Time}
\label{sec:modelling:parameter_fixing}
In the SEM model formulation, a fixing a parameter across time is effected by imposing an equality constraint on the
parameters to be estimated.
It is possible to quantify the validity of these parameter constraints in terms of the $\chi^2$ statistic.
Namely, when relaxing such a constraint, a single degree of freedom is won. The improvement (decrease) in the $\chi^2$
statistic can then be compared to a $\chi^2(1)$ distribution for statistical significance, where the null is
that the parameter constraint is appropriate.
When performing this test on the 180 constraints imposed in the model without mediators, the highest found test
statistic was for the first autoregressive parameter in wave 2023. It was $11.6$, which is greater than the critical
value of $3.84$. However, multiple testing ought to be considered.
When simply applying a Bonferroni correction for doing 180 tests, the critical value becomes $13.2$, thus none of the
imposed constraints are put into question.
While an alternative correction would maintain more power and might reject the null, especially because
the Bonferroni correction assumes the tests are independent but because the parameters are fixed across multiple
regressions this is clearly not true, it should also be noted that
the $\chi^2$ test's sensitivity to sample size means the null is rarely maintained. As such, the conclusion is drawn
that fixing the (regression) parameters across time is appropriate.
It is assumed that this finding generalises to the model with mediators.

However, the same cannot be said for the residual variances of $y$. Because of the way that SEM estimates the
parameters when regressands are also used as regressors (as in \cref{eq:methods:simultaneous_regression}),
the residual variance of MHI5 in 2023 is smaller than that the earlier waves. The model could be adjusted
to prevent this by setting the residual variances between MHI5-values that don't occur in the same regression as free
parameters (eliminating associated degrees of freedom). However, the residual variances are not of interest for the present
research question, so for simplicity's sake the residual variances of MHI5 are simply not constrained across time.
