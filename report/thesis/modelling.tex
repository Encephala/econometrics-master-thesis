\section{Regressors}
\label{sec:modelling:regressors}
As covariates, we include a basic set of sociodemographic variables available in the Background Variables of the LISS panel,
namely age, ethnicity, gender, marital status, education level, employment status and net household income.
While the first three are (largely) determined at birth, the last four might be hypothesised as mediators based on the literature
(\citeA{spiker2014mental,hjorth2016mental,frijters2014effect,thomson2022income} respectively).
However, all of these variables were found to vary so little in time in the dataset that including them as time-variant
covariates caused numerical instability, so they could only be included as time-invariants.
This precludes modelling them as mediators of the time-variant variable sports,
but the fact that they vary so little indicates that their effect as mediators could only be weak ($\zeta$ must be small),
so the error in estimating the total effect due to not modelling them as mediators could only be marginal.
Beyond this set, based on the literature and availability in the data, we include (self-reported) physical health
and presence of diagnosed diseases as additional mediators \cite{westcott2012resistance},
as well as BMI as a proxy for body-image \cite{westcott2012resistance}.
While the Health study also queries more nuanced variables like hospital admittance and medication use, these are
quite correlated with general health and disease status (e.g. $\rho \approx 0.55$ for medication use and disease status),
and are thus excluded for the sake of parsimony, noting the significant increase in complexity that mediators bring.
Both of these mediators are binary variables (after dummy encoding), so a logit regression would be preferred.
However, while \textit{lavaan} does support binary variables, it is not possible to use the MLR estimator nor FIML,
and crucially, numerical stability was poor when treating the mediators as binary variables.
Thus, the approximation is made that they are linear variables (i.e. variables on a ratio scale).
By virtue of the Taylor approximation, this approximation holds adequately if the estimated effect sizes of the regressors
are moderate.

Body composition, aerobic fitness and disease have been found to affect exercise adherence \cite{abernethy2012biophysical}.
Additionally, \citeA{ingledew2002effects} find BMI is a predictor of some exercise motives.
For all mediators thus, the reverse effect of the mediator on sports engagement is of concern.
Additionally, the reduced self-efficacy associated with poor mental health is a known cause for
eating disorders and general bad diet habits \cite{oellingrath2014eating}, influencing physical health.
The same mechanism also pertains to mediation through disease status, as diet is known to be a
significant determinant of immune function \cite{childs2019diet}, and naturally eating habits influence BMI.
The conclusion is therefore that for each of the three mediators,
reverse causality is of concern for both steps of the mediation action,
necessitating that only the lagged influence of regressors on regressands is ever considered.

Because only the long-term effect of sports on mental health is considered (recall \Cref{sec:methods:reverse_causality}),
it would be natural to use the cumulative years that someone has exercised as the variable of interest, rather than the
yearly exercise status. However, because it is not known how many years one has exercised at the time of joining this panel,
this variable can only be roughly approximated through a running sum. Additionally, information missing in one year would
complicate using the variable in all future years.
As such, sports is used directly, while examining different orders of the distributed lags to model effects beyond
one year ahead.

This study chooses to not include individual-specific effects $\alpha_i$ in the model. While they help justify
the ignorability assumption through eliminating unmeasured confounders, they also potentially obscure the effects of time-invariant mediators,
which equally invalidate the findings. The approach is thus instead to model individual-specific variability through the
time-invariant controls $z_i$. Additionally, inclusion of fixed effects entails estimating $N$ additional parameters
(the incidental parameter problem). Since it is likely that $N > \frac{p(p + 1)}{2}$, the model would not be identified.
Nothing prevents modelling random effects however, and in fact this is commonly done in SEM
(for instance \citeA{heck2001multilevel}). The author leaves it as an avenue for further research.

As a minor final comment, some dummy levels of controls were excluded, because they occurred so infrequently in the data
that including them as regressors led to numerical instability.
These were a non-binary gender (``other'' in the data), and perhaps surprisingly,
an annual household income between € 15\,000 and € 50\,000.

\section{Lag Selection}
\label{sec:modelling:lags}
For determining the autoregressive lag order $L_y$ and the distributed lag order $L_x$, cross validation was run as
described in \Cref{sec:methods:cv}.
However, for computational tractability, mediators were not modelled explicitly as such but simply included as controls.
This should have only a minor impact on forecasting MHI5, but makes a higher number of repeats $R = 50$ feasible.
It is assumed that the optimal model structure found in this way also applies when mediation regressions are also
included.
Additionally, it was found that only applying FIML to the endogenous variables yielded very similar parameter estimates
and thus similar forecasts. To then further improve computational tractability in cross validation, listwise deletion
was used for the exogenous variables (i.e. the controls).

Both $L_y$ and $L_x$ were varied from 1 to the highest number available based on the data.
Due to the missing Health study in 2014, this is $L_y = 8$, and because the sports has only been included since 2012,
$L_x = 11$.
The optimal $L_y$ is first estimated, as it has a more profound role in the model than the distributed lag.
When doing so, $L_x = 1$ is used as the most minimal model. When optimising $L_x$, the optimal set of AR lags is used.
It could technically be possible that a better model fit is found when not including all lags between the lowest
and the highest lag, for instance having the set of lags $\{1, 2, 4\}$. This was however not tested as it seems improbable
given the mechanisms at play, and thus would likely be overfitting.
\Cref{fig:modelling:cv_lags} visualises the found RMSPEs.

\begin{figure}[htbp]
    \centering
    \begin{subfigure}[t]{0.49\textwidth}
        \centering
        \includesvg[width=\textwidth]{figures/modelling/cv_AR.svg}
        \vspace{0.1em}
        \subcaption{Varying maximum autoregressive lag $L_y$}
    \end{subfigure}
    \hfill
    \begin{subfigure}[t]{0.49\textwidth}
        \centering
        \includesvg[width=\textwidth]{figures/modelling/cv_DL.svg}
        \vspace{0.1em}
        \subcaption{Varying maximum distributed lag $L_x$}
    \end{subfigure}
    \caption{Cross validated out-of-sample forecasting accuracy for varying lag orders, measured by root mean square error
    $\mu_{\text{RMSPE}}$.
    Error bars are standard deviations of accuracy across folds. The horizontal line represents the 1-$\sigma$ decision rule.
    The smaller red error bars represent uncertainty in $\mu_{\text{RMSPE}} + \sigma_{\text{RMSPE}}$}
    \label{fig:modelling:cv_lags}
\end{figure}

For the autoregressive lags, the most complex model with AR lag order $L_y = 8$ performs best. The decision rule
then indicates $L_y = 3$ is best, though the estimation uncertainty makes $L_y = 4$ also worth considering.
Recalling that the cross validation findings may not generalise as it is only done on the data for 2023, and noting that
simpler models generalise better, $L_y = 3$ is chosen.
For the distributed lags, while the prediction error trends downwards with increasing model complexity,
the model with $L_x = 1$ is clearly favoured.

It should be noted that both figures show that especially with more complex models, $\sigma_{\text{RMSPE}}$ and the estimation uncertainties
are significantly larger. This is likely evidence for some of the testing folds having erratic values.
While this can potentially be fixed by either stratifying testing folds on missingness,
experimenting with fold counts other than 5 or through increasing the number of repetitions $R$,
the impact of these outliers on the decision rule is minor and does not influence the outcome, so this was not done
in the end.

\section{Fixing Parameters across Time}
\label{sec:modelling:parameter_fixing}
In the SEM model formulation, a fixing a parameter across time is effected by imposing an equality constraint on the
parameters to be estimated.
It is possible to quantify the validity of these parameter constraints in terms of the $\chi^2$ statistic.
Namely, when relaxing such a constraint, a single degree of freedom is won. The improvement (decrease) in the $\chi^2$
statistic can then be compared to a $\chi^2(1)$ distribution for statistical significance, where the null is
that the parameter constraint is appropriate.
When performing this test on the 180 constraints imposed in the model without mediators, the highest found test
statistic was for the first autoregressive parameter in wave 2023. It was $11.6$, which is greater than the critical
value of $3.84$. However, multiple testing ought to be considered.
When simply applying a Bonferroni correction for doing 180 tests, the critical value becomes $13.2$, thus none of the
imposed constraints are put into question.
While an alternative correction would maintain more power and might reject the null, especially because
the Bonferroni correction assumes the tests are independent but because the parameters are fixed across multiple
regressions this is clearly not true, it should also be noted that
the $\chi^2$ test's sensitivity to sample size means the null is rarely maintained. As such, the conclusion is drawn
that fixing the regression parameters across time is appropriate.
It is assumed that this finding generalises to the model with mediators.

However, the same cannot be said for the residual variances of $y$. Because of the way that SEM estimates the
parameters when regressands are also used as regressors (as in \Cref{eq:methods:simultaneous_regression}),
the residual variance of MHI5 in 2023 is smaller than it is in the earlier waves. The model could be adjusted
to prevent this by setting the residual variances between MHI5-values that do not occur in the same regression as free
parameters (eliminating associated degrees of freedom). However, the residual variances are not of interest for the present
research question, so for simplicity's sake the residual variances of MHI5 are simply not constrained across time.

However, the same cannot be said for the residual variances of $y$.
These vary somewhat across the waves, and fixing them made model fit appreciably worse.
This can be attributed to the fact that a covariance matrix must be positive definite, so fixing diagonal
elements also imposes a restriction on associated off-diagonal elements in the same row (or equivalently
in the same column).
Because the residual variances are not of interest for the present research question, the variability of
the residual variance across time will not be examined further.

For the definitive models in \textit{lavaan} syntax, refer to the programming code (\Cref{chap:app:code}).
