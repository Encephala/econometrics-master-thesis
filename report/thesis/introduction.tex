The relationship between Physical Exercise (exercise) and mental wellbeing has been the topic of much recent literature,
both in empirical studies \cite{noetel2024effect, mahindru2023role}, and through mechanistic research \cite{smith2021role, lubans2016physical}.
The latter suggest various mechanisms for the positive influence of exercise on mental wellbeing.
Proposed mechanisms include neurobiological ones like structural changes to the brain and endorphin release (e.g. dopamine),
as well as psychosocial mechanisms like relatedness, improved body image and self-efficacy (confidence in own ability to perform activities),
and behavioural mechanisms like improved sleep, self-regulation and coping skills \cite{lubans2016physical}.
Note that these mechanisms can be categorised into short-term effects like endorphin release, and long-term effects
like improved body image; this will be crucial for the present analysis.

However, while cross-sectional studies consistently find a strong association between the two, \citeA{chekroud2018association}
note that the causal effect of exercise on mental health as studied in randomised controlled trials (RCT) has been inconsistent.
\citeA{chalder2012facilitated} for instance find an insignificant change in the Beck's Depression Inventory score
of $-0.54$ ($p = 0.68$), while \citeA{philippot2022impact} find a significant decrease in the Hospital Anxiety and
Depression Scale of $3.8$ ($p = 0.016$).
Due to this inconsistency, review articles only draw tentative conclusions, for example that exercise ``is probably [beneficial]
for psychiatric diseases'' \cite{peluso2005physical}, ``hold(s) promise in the treatment [...] of mental health conditions''
\cite{smith2021role}, et cetera.
Strikingly, the review of reviews on the topic by \citeA{biddle2011physical} concludes there is a distinct lack of good quality research.

I posit that observational studies to date have not been powered to draw conclusions about the causal effect
of exercise on mental health, in part because they have not explicitly modelled the reverse effect, namely that individuals with poor
mental health are less likely to engage in exercise. If this is true, estimation of the effect of exercise on mental health is
plagued by endogeneity, leading to inconsistent estimatioin of the causal effect, a fact that \citeA{leszczensky2022deal} note is often neglected in
longitudinal research.
There is a decided lack of literature on this reverse action, as is apparent in the review article by \citeA{fossati2021physical},
whose discussion of the forward effect is much more extensive than that of the reverse effect.
Nevertheless, they establish a consistent finding that mental health is associated with injury risk, which may hinder
one's ability to engage in exercise. Additionally, other research has found low mood and stress to be common barriers for
engaging in exercise \cite{firth2016motivating}.
\citeA{azevedo2012bidirectional} and \citeA{jerstad2010prospective} also find empirical evidence for this reverse relationship,
though other work is inconclusive \cite{birkeland2009longitudinal, ku2012physical}. Altogether, the assumption that engagement
in exercise is not influenced by mental health seems tenuous at best.

The aim of this work is then to remedy the discrepancy between on the one hand the consistent association found in
cross-sectional studies, and the inconsistent findings of trials on the other hand, by making sure the reverse effect
is accounted for.
Specifically, the research question is as follows:
\begin{quote}
    Is physical exercise an effective intervention for improving mental wellbeing?
\end{quote}
The emphasis is thus on the within-person effect of exercise, not simply the between-person correlation between the two.

Panel data provides the means through which the reciprocal relationship can be modelled. Specifically, the simple fact
that cause must preceed effect can be exploited with panel data.
\citeA{leszczensky2022deal} find excellent statistical properties of Structural Equation modelling (SEM)
when using panel data to model reciprocal action, namely lower bias and significantly greater efficiency than Arellano-Bond
(GMM) estimation (if the model is correctly specified).
Inspired by these findings, the present study follows the procedure outlined by \citeA{allison2017maximum}
for modelling reciprocal causation.
In SEM, arbitrary (linear) interactions amongst variables can be modelled, which gives a lot of freedom in model
specification. SEM thus lends itself well to studying a complex and nuanced topic like human psychology,
where manifold model assumptions might have to be made or relaxed.

The data studied is from the LISS panel, which is a representative (invite-only) panel of $7500$ Dutch individuals aged
sixteen and above \cite{scherpenzeel2010liss}. The panel comprises a broad range of questions, including the topics of exercise
and mental wellbeing, but also manifold other variables that can therefore be controlled for in the analysis.
Because the panel follows individuals for multiple years, it provides insight into the temporal nature of interactions
between exercise and mental health, which makes it possible to emphasise the within-person effect of exercise on mental health
rather than the between-person assocation between the two.

The present study aims to guide further research by contributing to a better understanding of the potentially
complicated interdependence between mental health and exercise.
